%%%%%%%%%%%%%%%%%%%%%%%%%%%%%%%%%%%%%%%%%%%%%%%%%%%%%%%%%%%%%%%
%
%     filename  = "YourName-Dissertation.tex",
%     version   = "1.3.0",
%     date      = "2013/10/21",
%     authors   = "Gary L. Gray,
%     copyright = "Gary L. Gray",
%     address   = "Engineering Science and Mechanics,
%                  212 Earth & Engineering Sciences Bldg.,
%                  Penn State University,
%                  University Park, PA 16802,
%                  USA",
%     telephone = "814-863-1778",
%     email     = "gray@psu.edu",
%
%%%%%%%%%%%%%%%%%%%%%%%%%%%%%%%%%%%%%%%%%%%%%%%%%%%%%%%%%%%%%%%
%
% Change History:
%
% 1.3.0	**	Added the cite package.
%
%		**	Give that the Graduate School now allows essentially
%			any line spacing, I have moved the line space setting
%			from psuthesis.cls to this driver file. Go ahead and
%			make it ugly if you want. :-)
%
%		**	Removed \addtocounter{page}{-1} after \psutitlepage is
%			executed. It made the paging of the frontmatter
%			incorrect. I can no longer remember why it was there.
%
%		**	Removed \psusigpage since the Graduate School now
%			provides the signature page.
%
%		**	Added the command \collegesubmittedto to add the College
%			in which the thesis/dissertation has been completed to
%			the title page.
%
%		**	Added instructions for documents that include a single
%			appendix since the Graduate School just hates calling it
%			``Appendix A'' is there is a single appendix.
%
%		**	Removed the fncychap package since I could not easily
%			find a way to make it work with documents that have a
%			single appendix.
%
%		**	Added the titlesec package so that the user can make the
%			format of the chapter titles a little less boring than
%			LaTeX's default.
%
% 1.2.2	**	Added some information to the main driver file (this
%			file) regarding the use of hyperref with the
%			psuthesis class. Thanks to Nathan Urban for pointing
%			out the included workaround.
%
% 1.2.1	**	Finally reproduced and fixed the problem where the
%			page number listed in the TOC for the Bibliography
%			was the last page number of the Bibliography.
%
%		**	Added 10pt and 11pt options to the document class,
%			though we have no idea why anyone would want to use
%			such insanely small font sizes since it will lead to
%			line lengths that are much too long.
%
% 1.2.0	**	Two additional class options have been added to
%			support honors theses submitted to the Schreyer
%			Honors College. These options are:
%			- honors
%			- honorsdepthead
%			See below for details.
%
%		**	We have also added the commands:
%			- honorsdegreeinfo
%			- honorsadviser
%			- honorsdepthead
%			Again, see below for details.
%
% 1.1.2	**	If you want to use the subfigure package with our
%			psuthesis class file, then you must must find the 
%			following line in the psuthesis.cls file:
%
%			\RequirePackage{tocloft}
%
%			and add the subfigure option. We have already set
%			this up for you in the psuthesis.cls file to make
%			this easy to do.
%
% 1.1.1	**	Added the fncychap package to the distribution.
%
% 1.1.0	**	The way that the thesis frontmatter and backmatter
%			is generated has been completely re-done in order
%			to be more intuitive.
%
%		**	We have added the ability to change the title of
%			the Dedication/Epigraph to anything you please.
%
%		**	In the process of changing the format of the Table
%			of Contents to conform to the inflexible rules of
%			the Grad School (the word ``Chapter'' and
%			``Appendix'' need to appear before the number and
%			letter, respectively), we have added an option to
%			the class called inlinechaptertoc that changes the
%			format of the Chapter/Appendix entries in the TOC.
%			Note that the tocloft package is now required.
%
%		**	Appendices should now start with the
%			\Appendix command rather than \chapter. See the
%			accompanying files for examples.
%
%		**	Added information regarding the Nontechnical
%			Abstract that is required of ESM students.
%
%		**	Added the fncychap package for those of you who like
%			the nice Chapter headings it provides. We like
%			Lenny, but you don't have to use it if you don't
%			want to. In addition, the other options are: Sonny,
%			Glenn, Conny, Rejne, and Bjarne
%
% 1.0.4	**	fixed the \addcontentsline entry for BibTeX within
%			the commented out text in the Bibliography section
%
% 1.0.3	**	added a sigpage option to conform to new Grad School
%			requirements
%
% 1.0.2	**	issued the \appendix command to start the appendices
%
%		**	moved the \addcontentsline for the bibliography so
%			that the bibliography now shows up on the right page
%			in the TOC
%
%		**	added some info if you use bibtex
%
% 1.0.1	**	eqlist and eqparbox are now included in the archive
%
%%%%%%%%%%%%%%%%%%%%%%%%%%%%%%%%%%%%%%%%%%%%%%%%%%%%%%%%%%%%%%%
%
% This is a template file to help get you started using the
% psuthesis.cls for theses and dissertations at Penn State
% University. You will, of course, need to put the
% psuthesis.cls file someplace that LaTeX will find it.
%
% We have set up a directory structure that we find to be clean
% and convenient. You can readjust it to suit your tastes. In
% fact, the structure used by our students is even a little
% more involved and commands are defined to point to the
% various directories.
%
% This document has been set up to be typeset using pdflatex.
% About the only thing you will need to change if typesetting
% using latex is the \DeclareGraphicsExtensions command.
%
% The psuthesis document class uses the same options as the
% book class. In addition, it requires that you have the
% ifthen, calc, setspace, and tocloft packages.
%
% The first additional option specifies the degree type. You
% can choose from:
%	Ph.D. using class option <phd>
%	M.S. using class option <ms>
%	M.Eng. using class option <meng>
%	M.A. using class option <ma>
%	B.S. using class option <bs>
%	B.A. using class option <ba>
%	Honors Baccalaureate using the option <honors>
%
% If you specify either ba or bs in addition to honors, it will
% just use the honors option and ignore the ba or bs option.
%
% The second additional option <inlinechaptertoc> determines
% the formatting of the Chapter entries in the Table of
% Contents. The default sets them as two-line entries (try it).
% If you want them as one-line entries, issue the
% inlinechaptertoc option.
%
% The class option ``honors'' should be used for theses
% submitted to the Schreyer Honors College. This option
% changes the formatting on the Title page so that the
% signatures appear on the Title page.
%
% The class option ``honorsdepthead'' adds the signature of the
% department head on the Title page for those baccalaureate
% theses that require this.
%
% The class option ``secondthesissupervisor'' should be used
% for baccalaureate honors degrees if you have a second
% Thesis Supervisor.
%
% The vita is only included with the phd option and it is
% placed at the end of the thesis. The permissions page is only
% included with the ms, meng, and ma options.
%%%%%%%%%%%%%%%%%%%%%%%%%%%%%%%%%%%%%%%%%%%%%%%%%%%%%%%%%%%%%%%
% Only one of the following lines should be used at a time.
%\documentclass[draft,phd,12pt]{psuthesis}
%\documentclass[draft,phd,inlinechaptertoc]{psuthesis}
%\documentclass[draft,ms]{psuthesis}
%\documentclass[draft,honorsdepthead,honors]{psuthesis}
\documentclass[phd,12pt]{psuthesis}
%\documentclass[draft,secondthesissupervisor,honors]{psuthesis}
%\documentclass[draft,bs]{psuthesis}

\usepackage[T1]{fontenc}
\usepackage{lmodern}
\usepackage{textcomp}
\usepackage{microtype}

%%%%%%%%%%%%%%%%%%%%%%%%%%%%
% Packages we like to use. %
%%%%%%%%%%%%%%%%%%%%%%%%%%%%
\usepackage{amsmath}
\usepackage{amssymb}
%\usepackage{amsthm}
%\usepackage{exscale}
%\usepackage[mathscr]{eucal}
%\usepackage{bm}
\usepackage{eqlist} % Makes for a nice list of symbols.
\usepackage[final]{graphicx}
\usepackage[dvipsnames]{color}
\DeclareGraphicsExtensions{.pdf, .jpg}

% http://www.tex.ac.uk/cgi-bin/texfaq2html?label=citesort
\usepackage{cite}

\usepackage{titlesec}

%%%%%%%%%%%%%%%%%%%%%%%%%%%%%%%
% Use of the hyperref package %
%%%%%%%%%%%%%%%%%%%%%%%%%%%%%%%
%
% This is optional and is included only for those students
% who want to use it.
%
% To the hyperref package, uncomment the following line:
%\usepackage{hyperref}
%
% Note that you should also uncomment the following line:
%\renewcommand{\theHchapter}{\thepart.\thechapter}
%
% to work around some a problem hyperref has with the fact
% the psuthesis class has unnumbered pages after which page
% counters are reset.

% Set the baselinestretch using the setspace package.
% The LaTeX Companion claims that a \baselinestretch of
% 1.24 gives one-and-a-half line spacing, which is allowed
% by the PSU thesis office. As of October 18, 2013, the Graduate
% School states ``The text of an eTD may be single-, double- or
% one- and-a-half-spaced.'' Go nuts!
\setstretch{1.24}


%%%%%%%%%%%%%%%%%%%%%%%%%%%%%%%%%%%%
% SPECIAL SYMBOLS AND NEW COMMANDS %
%%%%%%%%%%%%%%%%%%%%%%%%%%%%%%%%%%%%
% Place user-defined commands below.
% Roksana Added these
%Additional packages
\usepackage{paralist}
\usepackage{amsmath,amssymb,amsthm,mathtools}
%\usepackage{graphicx}
%\usepackage{bm}
\usepackage{xspace}
\usepackage{url}
%\usepackage{boxedminipage}
%\usepackage{wrapfig}
%\usepackage{ifthen}
\usepackage{color}
\usepackage{xcolor}
%\usepackage{framed}
%\usepackage{algorithmic,algorithm}
\usepackage{fullpage}
\usepackage{cite}
\usepackage{natbib}

%\usepackage{thmtools}
%\usepackage{thm-restate}
\usepackage[colorlinks, citecolor = black,
urlcolor=black,linkcolor=black,pdfstartview=FitH]{hyperref}
\usepackage[ruled, noend, noline]{algorithm2e}
\usepackage{verbatim}
\usepackage{bbm}

%Packages for appropriate appendix numbering
\usepackage{chngcntr}
\usepackage{apptools}

%Additional definitions and theorems
\newtheorem{theorem}{Theorem}[section]
\newtheorem{definition}[theorem]{Definition}
\newtheorem{conjecture}[theorem]{Conjecture}
\newtheorem{lemma}[theorem]{Lemma}
\newtheorem{claim}[theorem]{Claim}
\newtheorem{proposition}[theorem]{Proposition}
\newtheorem{fact}[theorem]{Fact}
\newtheorem{assumption}[theorem]{Assumption}
\newtheorem{observation}[theorem]{Observation}
%Numbering in appendix
\AtAppendix{\counterwithin{theorem}{section}}
%\AtAppendix{\counterwithin{claim}{section}}


\newcommand{\eps}{\varepsilon}

\def\hf{\hat{f}}
\def\hg{\hat{g}}
\def\Inf{{\sf Inf}}
\def\Viol{{\sf Viol}}
\def\I{{\mathbf I}}
\def\V{{\mathbf V}}
\def\R{{\mathbb R}}
\def\bb{\mathbf{b}}
\def\val{\mathtt{val}}
\newcommand{\ord}[2][th]{\ensuremath{{#2}^{\mathrm{#1}}}}
\newcommand{\coord}[2]{t^{#1}_{#2}}
\newcommand{\capt}{\mathtt{cap}}
\def\Yes{\mathbf{Yes}}
\def\No{\mathbf {No}}
\def\dec{\mathsf{val}}
\def\sgn{\mathsf{sgn}}
\def\supp{\mathsf{supp}}
\def\dd{{d'}}
\def\totcube{m}
\def\dir{\mathsf{dir}}

\newcommand{\dist}{\mathsf{dist}}

\newcommand{\abs}[1]{\left\lvert #1 \right\rvert}
\newcommand{\norm}[1]{\left\lVert #1 \right\rVert}
\newcommand{\ceil}[1]{\lceil#1\rceil}
\newcommand{\Exp}{\EX}
\newcommand{\floor}[1]{\lfloor#1\rfloor}
\newcommand{\cei}[1]{\lceil#1\rceil}
\newcommand{\EX}{\hbox{\bf E}}
\newcommand{\otilde}{\widetilde{O}}

\newcommand{\cA}{{\cal A}}
\newcommand{\cB}{\mathcal{B}}
\newcommand{\cC}{{\cal C}}
\newcommand{\cD}{\mathcal{D}}
\newcommand{\cE}{{\cal E}}
\newcommand{\cF}{\mathcal{F}}
\newcommand{\cG}{\mathcal{G}}
\newcommand{\cH}{{\cal H}}
\newcommand{\cI}{{\cal I}}
\newcommand{\cJ}{{\cal J}}
\newcommand{\cL}{{\cal L}}
\newcommand{\cM}{{\cal M}}
\newcommand{\cP}{\mathcal{P}}
\newcommand{\cQ}{\mathcal{Q}}
\newcommand{\cR}{{\cal R}}
\newcommand{\cS}{\mathcal{S}}
\newcommand{\cT}{{\cal T}}
\newcommand{\cU}{{\cal U}}
\newcommand{\cV}{{\cal V}}
\newcommand{\cX}{{\cal X}}

\newcommand{\N}{\mathbb N}
\newcommand{\calG}{{\cal G}}
\newcommand{\calA}{{\cal A}}
\newcommand{\calC}{{\cal C}}
\newcommand{\calE}{{\cal E}}


\newcommand{\bA}{\boldsymbol{A}}
\newcommand{\bD}{\boldsymbol{D}}
\newcommand{\bG}{\boldsymbol{G}}
\newcommand{\bR}{\boldsymbol{R}}
\newcommand{\bS}{\boldsymbol{S}}
\newcommand{\bT}{\boldsymbol{T}}
\newcommand{\bX}{\boldsymbol{X}}
\newcommand{\bY}{\boldsymbol{Y}}
\newcommand{\bZ}{\boldsymbol{Z}}

\newcommand{\Sec}[1]{\hyperref[sec:#1]{Section~\ref*{sec:#1}}} %section
\newcommand{\Eqn}[1]{\hyperref[eq:#1]{(\ref*{eq:#1})}} %equation
\newcommand{\Fig}[1]{\hyperref[fig:#1]{Fig.\,\ref*{fig:#1}}} %figure
\newcommand{\Tab}[1]{\hyperref[tab:#1]{Tab.\,\ref*{tab:#1}}} %table
\newcommand{\Thm}[1]{\hyperref[thm:#1]{Theorem~\ref*{thm:#1}}} %theorem
\newcommand{\Fact}[1]{\hyperref[fact:#1]{Fact\,\ref*{fact:#1}}} %fact
\newcommand{\Lem}[1]{\hyperref[lem:#1]{Lemma~\ref*{lem:#1}}} %lemma
\newcommand{\Prop}[1]{\hyperref[prop:#1]{Prop.~\ref*{prop:#1}}} %property
\newcommand{\Cor}[1]{\hyperref[cor:#1]{Corollary~\ref*{cor:#1}}} %corollary
\newcommand{\Conj}[1]{\hyperref[conj:#1]{Conjecture~\ref*{conj:#1}}} %conjecture
\newcommand{\Def}[1]{\hyperref[def:#1]{Definition~\ref*{def:#1}}} %definition
\newcommand{\Alg}[1]{\hyperref[alg:#1]{Algorithm~\ref*{alg:#1}}} %algorithm
\newcommand{\Ex}[1]{\hyperref[ex:#1]{Ex.~\ref*{ex:#1}}} %example
\newcommand{\Clm}[1]{\hyperref[clm:#1]{Claim~\ref*{clm:#1}}} %example
\newcommand{\Step}[1]{\hyperref[step:#1]{Step~\ref*{step:#1}}} %example

\newcommand{\new}[1]{{\color{red}{#1}}}

%%%%%%%%%%%%%%%%%%%%%%%%%%%%%%% End Roksana


%%%%%%%%%%%%%%%%%%%%%%%%%%%%%%%%%%%%%%%%%
% Renewed Float Parameters              %
% (Makes floats fit better on the page) %
%%%%%%%%%%%%%%%%%%%%%%%%%%%%%%%%%%%%%%%%%
\renewcommand{\floatpagefraction}{0.85}
\renewcommand{\topfraction}      {0.85}
\renewcommand{\bottomfraction}   {0.85}
\renewcommand{\textfraction}     {0.15}

% ----------------------------------------------------------- %

%%%%%%%%%%%%%%%%
% FRONT-MATTER %
%%%%%%%%%%%%%%%%
% Title
\title{Thesis Title}

% Author and Department
\author{Your Name}
\dept{Your Department Name}
% the degree will be conferred on this date
\degreedate{May 2013}
% year of your copyright
\copyrightyear{2013}

% This command is used for students submitting a thesis to the
% Schreyer Honors College. The argument of this command should
% contain every after the word ``requirements'' that appears on
% the title page. This provides the needed flexibility for
% all the degree types.
\honorsdegreeinfo{for a baccalaureate degree \\ in Engineering Science \\ with honors in Engineering Science}

% This is the document type. For example, this could also be:
%	Comprehensive Document
%	Thesis Proposal
%\documenttype{Thesis}
\documenttype{Dissertation}
%\documenttype{Comprehensive Document}


% This will generally be The Graduate School, though you can
% put anything in here to suit your needs.
\submittedto{The Graduate School}

% This is the college to in which you are submitting the
% thesis/dissertation.
\collegesubmittedto{College of Engineering}


%%%%%%%%%%%%%%%%%%
% Signatory Page %
%%%%%%%%%%%%%%%%%%
% You can have up to 7 committee members, i.e., one advisor
% and up to 6 readers.
%
% Begin by specifying the number of readers.
\numberofreaders{4}

% For baccalaureate honors degrees, enter the name of your
% honors adviser below.
\honorsadviser{Honors P. Adviser}

% For baccalaureate honors degrees, if you have a second
% Thesis Supervisor, enter his or her name below.
\secondthesissupervisor{Second T. Supervisor}

% For baccalaureate honors degrees, certain departments
% (e.g., Engineering Science and Mechanics) require the
% signature of the department head. In that case, enter the
% name of your department head below.
\honorsdepthead{Department Q. Head}

% Input reader information below. The optional argument, which
% comes first, goes on the second line before the name.
\advisor[Thesis Advisor, Chair of Committee]
		{Joseph H. Blow}
		{Professor of SomeThing}

\readerone[Optional Title Here]
			{Reader Name}
			{Professor of SomeThing}

\readertwo[Optional Title Here]
			{Reader Name}
			{Professor of SomeThing}

\readerthree[Optional Title Here]
			{Reader Name}
			{Professor of SomeThing}

\readerfour[Optional Title Here]
			{Reader Name}
			{Professor of SomeThing}

\readerfive[Optional Title Here]
			{Reader Name}
			{Professor of SomeThing}

% Format the Chapter headings using the titlesec package.
% You can format section headings and the like here too.
\definecolor{gray75}{gray}{0.75}
\newcommand{\hsp}{\hspace{15pt}}
\titleformat{\chapter}[display]{\fontsize{30}{30}\selectfont\bfseries\sffamily}{Chapter \thechapter\hsp\textcolor{gray75}{\raisebox{3pt}{|}}}{0pt}{}{}

\titleformat{\section}[block]{\Large\bfseries\sffamily}{\thesection}{12pt}{}{}
\titleformat{\subsection}[block]{\large\bfseries\sffamily}{\thesubsection}{12pt}{}{}


% Makes use of LaTeX's include facility. Add as many chapters
% and appendices as you like.
\includeonly{%
Chapter-1/Chapter-1,%
Chapter-2/Chapter-2,%
Chapter-3/Chapter-3,%
Chapter-4/Chapter-4,%
Chapter-5/Chapter-5,%
Appendix-A/Appendix-A,%
Appendix-B/Appendix-B,%
Appendix-C/Appendix-C,%
Appendix-D/Appendix-D,%
Appendix-E/Appendix-E%
}

\usepackage{listings}
%%%%%%%%%%%%%%%%%
% THE BEGINNING %
%%%%%%%%%%%%%%%%%
\begin{document}
%%%%%%%%%%%%%%%%%%%%%%%%
% Preliminary Material %
%%%%%%%%%%%%%%%%%%%%%%%%
% This command is needed to properly set up the frontmatter.
\frontmatter

%%%%%%%%%%%%%%%%%%%%%%%%%%%%%%%%%%%%%%%%%%%%%%%%%%%%%%%%%%%%%%
% IMPORTANT
%
% The following commands allow you to include all the
% frontmatter in your thesis. If you don't need one or more of
% these items, you can comment it out. Most of these items are
% actually required by the Grad School -- see the Thesis Guide
% for details regarding what is and what is not required for
% your particular degree.
%%%%%%%%%%%%%%%%%%%%%%%%%%%%%%%%%%%%%%%%%%%%%%%%%%%%%%%%%%%%%%
% !!! DO NOT CHANGE THE SEQUENCE OF THESE ITEMS !!!
%%%%%%%%%%%%%%%%%%%%%%%%%%%%%%%%%%%%%%%%%%%%%%%%%%%%%%%%%%%%%%

% Generates the title page based on info you have provided
% above.
\psutitlepage

% Generates the committee page -- this is bound with your
% thesis. If this is an baccalaureate honors thesis, then
% comment out this line.
\psucommitteepage

% Generates the abstract. The argument should point to the
% file containing your abstract. 
\thesisabstract{SupplementaryMaterial/Abstract}

% Generates the Table of Contents
\thesistableofcontents

% Generates the List of Figures
\thesislistoffigures

% Generates the List of Tables
\thesislistoftables

% Generates the List of Symbols. The argument should point to
% the file containing your List of Symbols. 
\thesislistofsymbols{SupplementaryMaterial/ListOfSymbols}

% Generates the Acknowledgments. The argument should point to
% the file containing your Acknowledgments. 
\thesisacknowledgments{SupplementaryMaterial/Acknowledgments}

% Generates the Epigraph/Dedication. The first argument should
% point to the file containing your Epigraph/Dedication and
% the second argument should be the title of this page. 
\thesisdedication{SupplementaryMaterial/Dedication}{Dedication}



%%%%%%%%%%%%%%%%%%%%%%%%%%%%%%%%%%%%%%%%%%%%%%%%%%%%%%
% This command is needed to get the main part of the %
% document going.                                    %
%%%%%%%%%%%%%%%%%%%%%%%%%%%%%%%%%%%%%%%%%%%%%%%%%%%%%%
\thesismainmatter

%%%%%%%%%%%%%%%%%%%%%%%%%%%%%%%%%%%%%%%%%%%%%%%%%%
% This is an AMS-LaTeX command to allow breaking %
% of displayed equations across pages. Note the  %
% closing the "}" just before the bibliography.  %
%%%%%%%%%%%%%%%%%%%%%%%%%%%%%%%%%%%%%%%%%%%%%%%%%%
\allowdisplaybreaks{
%\pagestyle{fancy}
%\fancyhead{}
%
%%%%%%%%%%%%%%%%%%%%%%
% THE ACTUAL CONTENT %
%%%%%%%%%%%%%%%%%%%%%%
% Chapters
%SourceDoc ../YourName-Dissertation.tex
\vspace*{-80mm}
\chapter{Introduction} \label{chapter1:introduction}

\section{Problem Definition}
%\paragraph{What is unateness?}

We study the problem of testing whether a given real-valued function $f$ on domain $[n]^d$, where $n,d\in\N,$ is unate.
A function $f:[n]^d \to \R$ is {\em unate} if for every coordinate $i\in [d]$, the function is either nonincreasing in the $\ord{i}$ coordinate or nondecreasing in the $\ord{i}$ coordinate. Monotone functions are special case of Unate functions, which are nondecreasing in all coordinates. 
%$\bb$-monotone functions, which have a particular direction in each coordinate (either nonincreasing or nondecreasing), specified by a bit-vector $\bb\in \{0,1\}^d$. More precisely,
%Given a bit-vector $\bb\in \{0,1\}^d$ call
% a function is $\bb$-monotone if it is nondecreasing in coordinates $i$ with $\bb_i = 0$ and nonincreasing in the other coordinates. Observe that a function $f$ is unate iff there exists some $\bb\in \{0,1\}^d$ for which $f$ is $\bb$-monotone.

The domain $[n]^d$ is called a {\em hypergrid} and the special case $\{0,1\}^d$ is called a {\em hypercube}. The {\sf distance} between two functions $f,g:[n]^d\rightarrow \mathbb{R}$ is equal to the fraction of points $x\in [n]^d$ where $f(x)\neq g(x)$. Given a parameter $\eps \in(0,1)$, two functions $f$ and $g$ are $\eps$-far from each other if the distance between $f$ and $g$ is at least $\eps$. A function $f$ is $\eps$-far from a property $P$ if it is $\eps$-far from any function which has property $P$. A {\em property tester}~\cite{GGR98,RS96} for a property $P$ is a randomized algorithm which, given parameter $\eps \in(0,1)$ and oracle access to the input function $f$, accepts $f$ with probability $\frac{2}{3}$, if it has the property $P$, and rejects $f$ with probability $\frac{2}{3}$, if it is $\eps$-far from $P$. A testing algorithm for property $P$ has {\em $1$-sided error} if it always accepts all input functions that satisfy  property $P$ and {\em $2$-sided error}, otherwise. A tester is {\em nonadaptive} if it makes all queries in advance, and {\em adaptive} if it can make queries after seeing answers to previous ones.

\section{Previous Works}
The problem of testing unateness was first considered by \citet{GGLRS00}. In their work, by extending their monotonicity tester, they obtain a nonadaptive, $1$-sided error tester for unateness with query complexity $O(\frac{d^{3/2}}{\epsilon})$. ~\citet{KS16} improved this upper bound by giving an adaptive unateness tester with query complexity $O(\frac{d\log d}{\eps})$. 

The related properties of monotonicity, Lipschit~\cite{JR13,CS13,BlaRY14} and bounded-derivative properties~\cite{CDJS17} has been studied extensively for different types of functions in the context of property testing. The problem of testing monotonicity of Boolean functions over the hypercube domain was first introduced by \citet{GGLRS00}. It was shown in \cite{DGLRRS99,GGLRS00} that monotonicity for hypercube domain can be tested with query complexity $O(\frac{d}{\epsilon})$. A monotonicity tester with better query complexity of $\tilde{O}(\frac{d^{7/8}}{\epsilon^{3/2}})$ was introduced by \citet{CS13b}. \citet{CST14} modified the tester of \citet{CS13b} and improved the query complexity to $\tilde{O}(\frac{d^{5/6}}{\epsilon^4})$. Most recently, \citet{KMS15} improved the query complexity of the tester to $\tilde{O}(\frac{\sqrt{d}}{\epsilon^2})$. The lower bound for any nonadaptive one-sided error tester for monotonicity over the hypercube was proved to be $\Omega(\sqrt{d})$ by \citet{FLNRRS02}. Also, \citet{CDST15} gave a lower bound of almost $\Omega(\sqrt{d})$ for any nonadaptive, two-sided error tester. Moreover, there is a lower bound of $\Omega(\min\{d, |R|^2\})$ over the real-valued functions by \citet{BBM12}. Most recently, \citet{BB16} gave a lower bound of $\tilde{\Omega}(d^{\frac{1}{4}})$ for adaptive testers of Boolean functions over the hypercube. \citet{CS13} proved that any adaptive, two-sided monotonicity tester for functions $f:[n]^d\rightarrow \mathbb{N}$ must make $\Omega(\frac{d\log n- \log \eps^{-1}}{\eps})$ queries.

%Testing of various properties of functions, including
%monotonicity
%(see, e.g., \cite{GGLRS00,DGLRRS99,EKKRV00,LR01,FLNRRS02,Fis04,HK07,AC06,HK08,BRW05,BBM12,BGJRW12,BCGM12,CS13, BlaRY14,BerRY14,CS14,CS16,CDJS17,CST14,CDST15,KMS15,BB15,BB16,DRTV16,PRV17} and recent surveys~\cite{Ras16,C16a}),
%the Lipschitz property \cite{JR13,CS13,BlaRY14}, bounded-derivative properties~\cite{CDJS17}, and unateness~\cite{GGLRS00,KS16}, has been studied extensively over the past two decades. Even though unateness testing was initially discussed in the seminal paper
%by Goldreich et al.~\cite{GGLRS00} that gave first testers for properties of functions,  relatively little is known about testing this property. All previous work on unateness testing focused on the special case of Boolean functions on domain $\{0,1\}^d$. The domain $\{0,1\}^d$ is called the {\em hypercube} and the more general domain $[n]^d$ is called the {\em hypergrid}.
%Goldreich et al.~\cite{GGLRS00}
%provided a $O(\frac{d^{3/2}}{\eps})$-query nonadaptive tester for unateness of Boolean functions on the hypercube.
%Recently, Khot and Shinkar~\cite{KS16} improved the query complexity
%to $O(\frac{d\log d}\eps)$, albeit with an adaptive tester.

\section{Our Work}

In this work, we improve previous results for unateness testing.

Specifically, we show that unateness of real-valued functions on hypercubes can be tested nonadaptively with $O(\frac d \eps \log \frac d \eps)$ queries and adaptively with $O(\frac d \eps)$ queries. More generally, we describe a $O(\frac d \eps \cdot(\log\frac d\eps + \log n))$-query nonadaptive tester and a
$O(\frac{d\log n}\eps)$-query adaptive tester of unateness of real-valued functions over hypergrids.

In contrast to the state of knowledge for unateness testing, the complexity of testing monotonicity of real-valued functions over the hypercube and the hypergrid has been resolved. For constant distance parameter $\eps$, it is known to be $\Theta(d\log n)$. Moreover, this bound holds for all {\em bounded-derivative} properties~\cite{CDJS17}, a large class that includes $\bb$-monotonicity and some properties quite different from monotonicity, such as the Lipschitz property. Amazingly, the upper bound for all these properties is achieved by the same simple and, in particular, nonadaptive, tester.
Even though proving lower bounds for adaptive testers has been challenging in general, a line of work, starting from Fischer~\cite{Fis04} and including \cite{BBM12,CS14,CDJS17}, has established that adaptivity does not help for this large class of properties. Since unateness is so closely related, it is natural to ask whether the same is true for testing unateness.

We answer this in the negative: we prove that any nonadaptive tester of real valued functions over the hypercube (for some constant distance parameter) must make $\Omega(d\log d)$ queries.
More generally, it needs $\Omega(d(\log d+\log n))$ queries for the hypergrid domain. These lower bounds complement our algorithms, completing the picture for unateness testing of real-valued functions.
From a property testing standpoint, our results establish that unateness is different from
monotonicity and, more generally, any derivative-bounded
property.

\section{Formal Statements and Technical Overview}
Our testers are summarized in the following theorem, stated for functions over the hypergrid domains. (Recall that the hypercube is a special case of the hypergrid with $n=2$.)

\begin{theorem}\label{thm:main-hg}
	Consider functions $f:[n]^d \to \R$ and a distance parameter $\eps\in (0,1/2)$.
	\begin{compactenum}
		\item\label{item:nonadaptive} There is a nonadaptive unateness tester that makes $O(\frac d \eps(\log\frac d\eps + \log n))$ queries\footnote{\scriptsize For many properties, when the domain is extended from the hypercube to the hypergrid, testers incur an extra multiplicative factor of $\log n$ in the query complexity. This is the case for our adaptive tester.
			 However, note that the complexity of nonadaptive unateness testing (for constant $\eps$) is $\Theta(d(\log d + \log n))$ rather than $\Theta(d\log d\log n).$}.
		
		\item\label{item:adaptive} There is an adaptive unateness tester  that makes $O(\frac{d\log n}\eps)$ queries.
	\end{compactenum}
	Both testers have one-sided error.
\end{theorem}

\noindent
Our main technical contribution is the proof that the extra $\Omega(\log d)$ is needed for nonadaptive testers.
This result demonstrates a gap between adaptive and nonadaptive unateness testing.
\begin{theorem}\label{thm:non-adap-lb-1}
	Any nonadaptive unateness tester (even with two-sided error) for real-valued functions $f:\{0,1\}^d \to \R$ with distance parameter $\eps = 1/8$  must make $\Omega(d\log d)$ queries.
\end{theorem}

\noindent The lower bound for adaptive testers is an easy adaptation of the monotonicity lower bound in~\cite{CS14}. 
We state this theorem for completeness and prove it in Appendix~\ref{sec:adap-lb}.
\begin{theorem}\label{thm:adap-lb}
Any unateness tester  for functions $f:[n]^d \to \R$ with distance parameter $\eps \in (0,1/4)$ must make $\Omega\left(\frac{d \log n}{\eps}-\frac{\log 1/\eps}{\eps}\right)$ queries.
\end{theorem}
Theorems~\ref{thm:non-adap-lb-1} and \ref{thm:adap-lb} directly imply that our nonadaptive tester is optimal for constant $\eps$, even for the hypergrid domain. The details appear in Appendix~\ref{sec:na-lb-hg}.

\subsection{Overview of Techniques}\label{sec:intro-tech}
We first consider the hypercube domain. For each $i\in[d],$ an {\em $i$-edge} of the hypercube is a pair $(x,y)$ of points in $\{0,1\}^d$, where $x_i=0,y_i=1$, and $x_j=y_j$ for all $j\in([d] \setminus \{i\})$. Given an input function $f:\{0,1\}^d\to\R$, we say an $i$-edge $(x,y)$ is {\em increasing} if $f(x)<f(y)$, {\em decreasing} if $f(x)>f(y),$ and {\em constant} if $f(x)=f(y)$.

Our nonadaptive unateness tester on the hypercube uses the work investment strategy from~\cite{BerRY14} (also refer to Section 8.2.4 of Goldreich's book~\cite{Go-book}) to ``guess'' a good dimension where to look for violations of unateness (specifically, both increasing and decreasing edges).  For all $i\in[d]$, let $\alpha_i$ be the fraction of the $i$-edges that are decreasing, $\beta_i$ be the fraction of the $i$-edges that are increasing, and $\mu_i = \min(\alpha_i,\beta_i)$. The dimension reduction theorem from~\cite{CDJS17} implies that if the input function is $\eps$-far from unate, then the average of $\mu_i$ over all dimensions is at least $\frac\eps{4d}$. If the tester knew which dimension had $\mu_i=\Omega(\eps/d)$, it could detect a violation with high probability by querying the endpoints of $O(1/\mu_i)=O(d/\eps)$ uniformly random edges.
However, the tester does not know which $\mu_i$ is large
and, intuitively, nonadaptively checks the following
$\log d$ different scenarios, one for each $k\in[\log d]$: exactly $2^k$ different
$\mu_i$'s are $\eps/2^k$, and all others are $0$. This leads to the query complexity of $O(\frac {d\log d}\eps).$

With adaptivity, this search  through $\log d$ different scenarios is not required.
A pair of queries in each dimension detects influential coordinates (i.e., dimensions with many non-constant edges), and the algorithm focuses
on finding violations among those coordinates. This leads to the query complexity of $O(d/\eps)$, removing the $\log d$ factor.

It is relatively easy to extend (both adaptive and nonadaptive) testers from hypercubes to hypergrids by incurring an extra factor of $\log n$ in the query complexity. The role of $i$-edges is now played by {\em $i$-lines}. An $i$-line is a set of $n$ domain points that differ only on coordinate $i$. The domain $[n]$ is called a line. Monotonicity on the line (a.k.a. sortedness) can be tested with $O(\frac{\log n}\eps)$ queries, using, for example, the classical {\em tree tester} from \cite{EKKRV00}. Instead of sampling a random $i$-edge, we sample a random $i$-line $\ell$ and run the tree tester on the restriction $f_{|\ell}$ of function $f$ to the line $\ell$.
This is optimal for adaptive testers, but, interestingly, not for nonadaptive testers.
We show that for each function $f$ on the line that is $\eps$-far from unateness, one of the two scenarios happen: (1) the tree tester is likely to find a violation of unateness; (2) function $f$ is increasing (and also decreasing) on a constant fraction of pairs in $[n]$. This new angle on the classical tester allows us to replace the factor $(\log d)(\log n)$ with $\log d + \log n$ in the query complexity.
Thus, the nonadaptive complexity becomes $O(d(\log d + \log n))$, which we show is optimal.

\subsection{The nonadaptive lower bound}
Our most significant finding is the $\log d$ gap in the query complexity between adaptive and nonadaptive testing of unateness.
By previous work~\cite{Fis04,CS14}, it suffices to prove lower bounds for {\em comparison-based} testers, i.e., testers that
can only perform comparisons of the function values at queried points, but cannot use the values themselves.
Our main technical contribution is the $\Omega(d\log d)$ lower
bound for nonadaptive comparison-based testers of unateness on hypercube domains.

Intuitively, we wish to construct $K=\Theta(\log d)$ families of functions where,
for each $k \in [K]$, functions in the $\ord{k}$ family have
$2^k$ dimensions $i$ with $\mu_i=\Theta(1/2^k)$, while $\mu_i=0$ for all other dimensions.
What makes the construction challenging is the existence of a \emph{single, universal} nonadaptive
$O(d)$-tester for all
$\bb$-monotonicity properties, proven in~\cite{CDJS17}. In other words, there is a single
distribution on  $O(d)$ queries that defines a nonadaptive property tester for
$\bb$-monotonicity, regardless of $\bb$.
Since unateness
is the union of all $\bb$-monotonicity properties,
our construction must be able to fool such algorithms.
Furthermore, nonadaptivity must be critical, since
we obtained a $O(d)$-query adaptive tester for unateness.

Another obstacle is that once a tester finds a non-constant edge in each dimension, the problem reduces to testing $\bb$-monotonicity for a vector $\bb$ determined by the directions (increasing or decreasing) of the non-constant edges. That is, intuitively, most edges in our construction must be constant. This is one of the main technical challenges. The previous lower bound constructions for monotonicity testing \cite{BBM12,CS14} crucially used the fact that all edges in the hard functions were non-constant.

We briefly describe how we overcome the problems mentioned above.
By Yao's minimax principle, it suffices to construct $\Yes$ and $\No$ distributions that
a deterministic nonadaptive tester cannot distinguish.
First, for some parameter $m$, we partition the hypercube into $\totcube$ subcubes based of the first $\log_2\totcube$ most significant coordinates.
Both distributions, $\Yes$ and $\No$, sample a uniform $k$ from $[K]$, where $K=\Theta(\log d)$, and a set $R\subseteq[d]$ of cardinality $2^k$.
Furthermore, each subcube $j\in [\totcube]$ selects an ``action dimension'' $r_j\in R$ uniformly at random. For both distributions, in any particular subcube $j$, the function value is completely determined by the coordinates {\em not in} $R$, and the random coordinate $r_j\in R$. Note that all the $i$-edges for $i\in (R\setminus \{r_j\})$ are constant.
Within the subcube, the function is a linear function
with exponentially increasing coefficients.
In the $\Yes$ distribution, any two cubes $j,j'$ with the same action dimension orient the edges in that dimension the same  way (both increasing or both decreasing), while in the $\No$ distribution each cube decides on the orientation independently.
The former correlation maintains unateness while the  latter independence creates distance to unateness.
We prove that to distinguish the distributions,
any comparison-based nonadaptive tester must find two distinct subcubes
with the same action dimension $r_j$ and, furthermore, make a specific
query (in both) that reveals the coefficient of $r_j$.
We show that, with $o(d\log d)$ queries, the probability of this event is negligible.

\chapter{Upper Bounds}\label{sec:upper-bounds}

In this section, we prove parts 1-2 of Theorem~\ref{thm:main-hg}, starting from the hypercube domain.

Recall the definition of $i$-edges and $i$-lines from Section~\ref{sec:intro-tech} and what it means for an edge to be increasing, decreasing, and constant.

The starting point for our algorithms is the dimension reduction theorem from~\cite{CDJS17}. It bounds the distance of $f:[n]^d \to \R$ to monotonicity in terms of average distances of restrictions of $f$ to one-dimensional functions.
%One-dimensional domain $[n]$ is called a {\em line}.
\begin{theorem}[Dimension Reduction, Theorem 1.8 in \cite{CDJS17}] \label{thm:dimred} Fix a bit vector ${\bf b}\in\{0,1\}^d$ and a function $f:[n]^d \to \R$ which is $\eps$-far from ${\bf b}$-monotonicity.
	For all $i\in[d]$, let $\mu_i$ be the average distance of $f_{|\ell}$ to $\bb_i$-monotonicity over all $i$-lines $\ell$.
	Then, $$\sum_{i=1}^d \mu_i \geq \frac{\eps}{4}.$$
\end{theorem}
For the special case of the hypercube domains, $i$-lines become $i$-edges, and the average distance $\mu_i$ to $\bb_i$-monotonicity is the fraction of $i$-edges on which the function is not $\bb_i$-monotone.


\section{The Nonadaptive Tester over the Hypercube}\label{sec:non-adap-ub}
We now describe Algorithm~\ref{alg:na-unate}, the nonadaptive tester for unateness over the hypercubes.

\begin{algorithm}
\caption{The Nonadaptive Unateness Tester over Hypercubes} \label{alg:na-unate}
\SetKwInOut{Input}{input}\SetKwInOut{Output}{output}
\SetKwFor{RepeatTimes}{repeat}{times}{end}
\SetKwFor{RepeatIterations}{repeat}{iterations}{end}
\Input{distance parameter $\eps \in (0,1/2)$; query access to a function $f:\{0,1\}^d \to \R$.}
\DontPrintSemicolon
\BlankLine
\nl \For{$r = 1$ to $\cei{3\log(4d/\eps)}$}{
\nl \RepeatTimes{$s_r = \cei{\frac{16d \ln 4}{\eps\cdot 2^r}}$}{
\nl \label{step:sample-dim}Sample a dimension $i \in [d]$ uniformly at random. \;
\nl \label{step:reject-hc}Sample $3 \cdot 2^r$ $i$-edges uniformly and independently at random and {\bf reject} if there exists an increasing edge and a decreasing edge among the sampled edges.\;
}
}
\nl {\bf accept}\;
\end{algorithm}

It is evident that \Alg{na-unate} is a nonadaptive, one-sided error tester. Furthermore,
its query complexity is $O\left(\frac{d}{\eps}\log\frac{d}{\eps}\right)$. It suffices to prove
the following.

\begin{lemma} \label{lem:na-unate-rej}
If $f$ is $\eps$-far from unate,
	\Alg{na-unate} rejects with probability at least $2/3$.
\end{lemma}

\begin{proof} Recall that $\alpha_i$ is the fraction of $i$-edges that are decreasing, $\beta_i$ is the fraction of $i$-edges that are increasing and $\mu_i = \min(\alpha_i,\beta_i).$
	
	
	Define the $d$-dimensional bit vector ${\bf b}$ as follows: for each $i\in [d],$ let ${\bf b}_i = 0$
	if $\alpha_i < \beta_i$ and $\bb_i = 1$ otherwise.
	Observe that the average distance of $f$ to $\bb_i$-monotonicity over a random $i$-edge is precisely $\mu_i$.
	Since $f$ is $\eps$-far from being unate,
	$f$ is also $\eps$-far from being ${\bf b}$-monotone.
	By \Thm{dimred}, $\sum_{i \in [d]} \mu_i \geq \frac\eps 4$. Hence, $\Exp_{i \in [d]}[\mu_i] \geq \frac\eps{4d}$.
	We now apply the work investment strategy due to Berman et al.~\cite{BerRY14} to get an upper bound on the probability that \Alg{na-unate} fails to reject.

\begin{theorem}[\cite{BerRY14}]\label{thm:wis}
For a random variable $X \in [0,1]$ with $\Exp[X] \geq \mu$ for $\mu < \frac 1 2$, let $p_r = \Pr[X \geq 2^{-r}]$ and $\delta \in (0,1)$ be the desired error probability. Let $s_r = \frac{4 \ln 1/\delta}{\mu \cdot 2^r}$. Then,
$$\prod\limits_{r=1}^{\lceil 3\log (1/\mu) \rceil} (1-p_r)^{s_r} \leq \delta. $$
\end{theorem}

\noindent Consider running \Alg{na-unate} on a function $f$ that is $\eps$-far from unate. Let $X = \mu_i$ where $i$ is sampled uniformly at random from $[d]$. Then $\Exp[X] \geq \frac\eps{4d}$. Applying the work investment strategy (\Thm{wis}) on $X$ with $\mu=\frac\eps{4d}$, we get that the probability that, in some iteration, \Step{sample-dim} samples a dimension $i$ such that $\mu_i\geq 2^{-r}$ is at least $1-\delta$.
We set $\delta = 1/4$.
Conditioned on sampling such a dimension, the probability that \Step{reject-hc} fails to obtain an increasing edge and a decreasing edge among its $3 \cdot 2^r$ samples is at most $2\left(1-2^{-r}\right)^{3 \cdot 2^r} \leq 2e^{-3} < 1/9$, as the fraction of both increasing and decreasing edges in the dimension is at least $2^{-r}$.
Hence, the probability that \Alg{na-unate} rejects $f$ is at least $\frac{3}{4} \cdot \frac{8}{9} = \frac{2}{3}$,
which completes the proof of Lemma~\ref{lem:na-unate-rej}.
\end{proof}

\section{The Adaptive Tester over the Hypercube}\label{sec:adap-ub}
We now describe Algorithm~\ref{alg:adap-unate}, an adaptive tester for unateness over the hypercube domain with good expected query complexity. The final tester is obtained by repeating this tester and accepting if the number of queries exceeds a specified bound.

\begin{algorithm}
\caption{The Adaptive Unateness Tester over Hypercubes} \label{alg:adap-unate}
\SetKwInOut{Input}{input}\SetKwInOut{Output}{output}
\SetKwFor{RepeatTimes}{repeat}{times}{end}
\SetKwFor{RepeatIterations}{repeat}{iterations}{end}
\Input{distance parameter $\eps \in (0,1/2)$; query access to a function $f:\{0,1\}^d \to \R$.}
\DontPrintSemicolon
\BlankLine
\nl\label{step:repeat}\RepeatTimes{$10/\eps$}{
\nl\label{step:begin-adap-hc}\For{$i = 1$ to $d$}{
\nl\label{step:sample-s}Sample an $i$-edge $e_i$ uniformly at random.\;
\nl\label{step:end-build-s}\If{$e_i$ is non-constant (i.e., increasing or decreasing)}{
\nl\label{step:stage-2-edge-sample} Sample $i$-edges uniformly at random till we obtain a non-constant edge $e_i'$.\;
\nl\label{step:end-adap-hc} {\bf reject} if one of the edges $e_i, e_i'$ is increasing and the other is decreasing.\;}
}}
\nl {\bf accept}
\end{algorithm}

\begin{claim} \label{clm:time}
The expected number of queries made by
\Alg{adap-unate} is $40d/\eps$.
\end{claim}

\begin{proof}
Consider one iteration of the {\bf repeat}-loop in Step~\ref{step:repeat}.
We prove that the expected number of queries in this iteration is $4d$.
The total number of queries in Step~\ref{step:sample-s} is $2d$, as 2 points per dimension are queried.
	Let $E_i$ be the event that edge $e_i$ is non-constant and $T_i$ be the random variable for the number of $i$-edges sampled in \Step{stage-2-edge-sample}. Then $\Exp[T_i] = \frac 1{\alpha_i+\beta_i}=\frac 1{\Pr[E_i]}$. Therefore, the expected number of all edges sampled in \Step{stage-2-edge-sample} is $\sum_{i=1}^d \Pr[E_i]\cdot\Exp[T_i]=\sum_{i=1}^d \Pr[E_i] \cdot \frac 1{\Pr[E_i]} = d$. Hence, the expected number of queries in \Step{stage-2-edge-sample} is $2d$. Since there are $10/\eps$ iterations in Step~\ref{step:repeat}, the expected number of queries in \Alg{adap-unate} is $40d/\eps$.
\end{proof}

\begin{claim} \label{clm:rej}
If $f$ is $\eps$-far from unate, \Alg{adap-unate} accepts with probability at most $1/6$.
\end{claim}

\begin{proof}
First, we bound the probability that a violation of unateness is detected in some dimension $i \in [d]$ in one iteration of the {\bf repeat}-loop.
Consider the probability of finding a %strictly
decreasing $i$-edge in Step~\ref{step:sample-s}, and an %strictly
increasing $i$-edge in \Step{stage-2-edge-sample}.
	The former is exactly $\alpha_i,$ and the latter is $\frac{\beta_i}{\alpha_i+\beta_i}$. Therefore, the probability we detect a violation from dimension $i$ is
$\frac{2 \alpha_i \beta_i}{\alpha_i+\beta_i} \geq \min(\alpha_i,\beta_i) = \mu_i$.
The probability that we fail to detect a violation in any of the $d$ dimensions is at most $\prod_{i =1}^d (1-\mu_i) \leq \exp\big(-\sum_{i = 1}^d \mu_i\big),$ which is at most $e^{-\eps/4}$ by \Thm{dimred} (Dimension Reduction). By Taylor expansion of $e^{-\eps/4}$, the probability of finding a violation in one iteration is at least $1-e^{-\eps/4} \geq
\frac{\eps}{4} - \frac{\eps^2}{32} >
\frac{\eps}{5}$. %as $\eps < 1/2$.
The probability that \Alg{adap-unate} does not reject in any iteration is at most $(1-\eps/5)^{10/\eps} < 1/6$.
\end{proof}

\begin{proof}[{Proof of Theorem~\ref{thm:main-hg}, Part 2} (for the special case of the hypercube domain)]
We run \Alg{adap-unate}, aborting and accepting if we ever make more than $240d/\eps$ queries. By Markov's inequality, the probability of aborting is at most $1/6$.
By Claim~\ref{clm:rej}, if $f$ is $\eps$-far from unate, \Alg{adap-unate} accepts with probability at most $1/6$. The theorem follows by a union bound.
\end{proof}

\section{Extension to Hypergrids}\label{sec:hypergrids}
We start by establishing terminology for lines and pairs.
Consider a function $f:[n]^d\to\R$.
Recall the definition of $i$-lines from Section~\ref{sec:intro-tech}.
A pair of points that differ only in coordinate $i$ is called an $i$-pair.
An $i$-pair $(x,y)$ with $x_i < y_i$ is called {\em increasing} if $f(x) < f(y)$, {\em decreasing} if $f(x) > f(y),$ and {\em constant} if $f(x) = f(y)$.

\begin{algorithm}
\caption{Tree Tester}\label{alg:tree-tester}
\SetKwInOut{Input}{input}\SetKwInOut{Output}{output}
\SetKwFor{RepeatTimes}{repeat}{times}{end}
\SetKwFor{RepeatIterations}{repeat}{iterations}{end}
\Input{Query access to a function $h:[n] \mapsto \R$.}
\DontPrintSemicolon
\BlankLine

\nl Pick $x \in [n]$ uniformly at random. \;
\nl Let $Q_x\subseteq [n]$ be the set of points visited in a binary search for $x$. Query $h$ on all points in $Q_x$. \;
\nl If there is an increasing pair in $Q_x$, set $\dir \gets \{\uparrow\}$; otherwise, $\dir\gets\emptyset.$\;
\nl If there is a decreasing pair in $Q_x$, update $\dir \gets\dir \cup \{\downarrow\}$. \;
\nl {\bf Return} $\dir$.
\end{algorithm}

The main tool for extending Algorithms~\ref{alg:na-unate} and \ref{alg:adap-unate} to work on hypergrids is the {\em tree tester}, designed by Ergun et al.~\cite{EKKRV00} to test monotonicity of functions $h:[n]\to\R$.
We modify the tree tester to return information about directions it observed instead of just accepting or rejecting. See Algorithm~\ref{alg:tree-tester}. 
We call a function $h:[n] \mapsto \R$ \emph{antimonotone} if $f(x) \geq f(y)$ for all $x < y$.
The following lemma summarizes the guarantee of the tree tester.

\begin{lemma}[\cite{EKKRV00, CDJS17}] \label{lem:line}
If $h:[n] \mapsto \R$ is $\eps$-far from monotone (respectively, antimonotone), then the output of \Alg{tree-tester} on $h$ contains $\downarrow$ (respectively, $\uparrow$) with probability at least $\eps$.
\end{lemma}

\begin{algorithm}
\caption{The Adaptive Unateness Tester over Hypergrids} \label{alg:adap-unate-hg}
\SetKwInOut{Input}{input}\SetKwInOut{Output}{output}
\SetKwFor{RepeatTimes}{repeat}{times}{end}
\SetKwFor{RepeatIterations}{repeat}{iterations}{end}
\Input{distance parameter $\eps \in (0,1/2)$; query access to a function $f:[n]^d \to \R$.}
\DontPrintSemicolon
\BlankLine
\nl \RepeatTimes{$10/\eps$}{
\nl\label{step:begin-adap-hg}\For{$i = 1$ to $d$}{
\nl Sample an $i$-line $\ell_i$ uniformly at random.\;
\nl\label{step:test-i-line}Let $\dir_i$ be the output of \Alg{tree-tester} on $f_{|\ell_i}$. \;
\nl \If{$\dir_i \ne \emptyset$}{
\nl\label{step:verify-i-line} Sample $i$-lines uniformly at random and run \Alg{tree-tester} on $f$ restricted to each line until it returns a non-empty set. Call it $\dir_i'$.\;
\nl  If $\dir_i \cup \dir_i' = \{\uparrow, \downarrow \}$, {\bf reject}.\;
}}}
\nl {\bf accept}
\end{algorithm}

\begin{algorithm}[h!]
	\caption{The Nonadaptive Unateness Tester over Hypergrids} \label{alg:na-unate-hg}
	\SetKwInOut{Input}{input}\SetKwInOut{Output}{output}
	\SetKwFor{RepeatTimes}{repeat}{times}{end}
	\SetKwFor{RepeatIterations}{repeat}{iterations}{end}
	\Input{distance parameter $\eps \in (0,1/2)$; query access to a function $f:[n]^d \to \R$.}
	\DontPrintSemicolon
	\BlankLine
	
	\nl \label{step:mono}\RepeatTimes{$220/\eps$}{
		\nl \label{step:mono-start}\For{$i=1$ to $d$}{
			\nl Sample an $i$-line $\ell$ uniformly at random. \;
			\nl \label{step:log}{\bf Reject} if \Alg{tree-tester}, on input $f_{|\ell}$, returns $\{\uparrow,\downarrow\}$.
		}
	}
	\nl \label{step:cube}\For{$r=1$ to $\cei{3 \log(200d/\eps)}$}{
		\nl \RepeatTimes{$s_r = \cei{\frac{800d \ln 4}{\eps \cdot 2^r}}$}{
			\nl \label{step:samp-hg1}Sample a dimension $i \in [d]$ uniformly at random.\;
			\nl \label{step:tree-tester1}Sample $3 \cdot 2^r$ $i$-pairs uniformly and independently at random.\;
			\nl \label{step:rej-hg1}If we find an increasing and a decreasing pair among the sampled pairs, {\bf reject}.\;
		}
	}
	\nl {\bf accept}			
\end{algorithm}

Our hypergrid testers are stated in Algorithms~\ref{alg:adap-unate-hg} and~\ref{alg:na-unate-hg}.
Next, we explain how Lemma~\ref{lem:line} and Theorem~\ref{thm:dimred} are used in the analysis of the adaptive tester.
For a dimension $i\in[d]$, let $\alpha_i$ and $\beta_i$ denote the average distance of $f_{|\ell}$ to monotonicity and antimonotonicity, respectively, over all $i$-lines $\ell$.
Then $\mu_i := \min(\alpha_i,\beta_i)$ is the average fraction of points per $i$-line that needs to change to make $f$ unate.
Define the $\bb$-vector with $\bb_i = 0$ if $\alpha_i < \beta_i$,
and $\bb_i = 1$ otherwise.
By \Thm{dimred},
if $f$ is $\eps$-far from unate, and thus $\eps$-far from $\bb$-monotone, then $\sum_{i=1}^d \mu_i \geq \eps/4$.
By \Lem{line}, the probability that the output of \Alg{tree-tester} on $f_{|\ell}$
contains $\downarrow$ (respectively, $\uparrow$), where $\ell$ is a uniformly random $i$-line,
is at least $\alpha_i$ (respectively, $\beta_i$).
The rest of the analysis of Algorithm~\ref{alg:adap-unate-hg} is similar to that in the hypercube case.

\begin{proof}[Proof of Theorem~\ref{thm:main-hg}, Part 2]
	The tester is Algorithm~\ref{alg:adap-unate-hg}.
As in the proof of Claim~\ref{clm:time}, the expected running time of Algorithm~\ref{alg:adap-unate-hg} is at most $(40d\log n)/\eps$.
The proof of Claim~\ref{clm:rej} carries over almost word-to-word. Fix dimension $i$. The probability that $\downarrow \in \dir_i$ in Step~\ref{step:test-i-line} is at least $\alpha_i$.
The probability that $\uparrow \in \dir_i'$ in Step~\ref{step:verify-i-line} is at least $\frac{\beta_i}{\alpha_i + \beta_i}$.
The rest of the calculation is identical
to that of the proof of \Clm{rej}.
\end{proof}

To analyze the nonadaptive tester,
we prove Lemma~\ref{lem:treetester}, which demonstrates the power of the tree tester and  may be of independent interest.

\begin{lemma}\label{lem:treetester}
	Consider a function $h:[n]\to \R$ which is $\eps$-far from monotone (respectively, antimonotone). At least one of the following holds:
	\begin{compactenum}
		\item $\Pr[\text{\Alg{tree-tester}, on input $h$, returns }\{\uparrow,\downarrow\}] \geq \eps/25$.%where the probability is over the sample $Q$ drawn by the tree tester.
		\item $\Pr_{u,v \in [n]}[(u,v) \textrm{ is a decreasing (respectively, increasing)  pair}] \geq \eps/25$.
	\end{compactenum}
\end{lemma}


\begin{proof}
\def\lca{\mathsf{lca}}
\def\updown{$\{\uparrow, \downarrow \}$}
Let $T$ be a balanced binary search tree consisting of elements in $[n]$, such that the set of points visited in a binary search for some $x \in [n]$ corresponds to a path from the root to the node containing $x$ in $T$.
Let $Q_x$ denote the set of points visited in a binary search for $x \in [n]$.
For $x, y \in [n]$, denote the least common ancestor of $x$ and $y$ by $\lca(x,y)$.

Let $W_{\uparrow \downarrow}$ be a set of points $x$ such that $Q_x$ contains both an increasing and a decreasing pair (with respect to $h$). If $|W_{\uparrow \downarrow}|\geq \frac{\eps n}{10}$, then Case 1 of Lemma~\ref{lem:treetester} holds. We may therefore assume that $|W_{\uparrow \downarrow}|< \frac{\eps n}{10}$.
Let $\calE$ be the event that for any $u,v \in [n]$ such that $u < v$, the pair $(u,v)$ is decreasing. We will prove
that $\Pr[\calE] \geq \eps/25$.

Let $W_{\downarrow}$ be that set of points $x \in [n]$ such that $Q_x$
contains a decreasing pair. Similarly, define the set $W_{\uparrow}$. Let $W_c$ denote the set of points $x$ such that $h_{|Q_x}$ is constant.

\begin{claim}[\cite{EKKRV00}]\label{clm:2}
	The function $h$ restricted to the set $W_{\uparrow}\cup W_c$ is monotone.
\end{claim}

\begin{proof}
The proof is by contradiction. Suppose $x,y \in (W_{\uparrow} \cup W_c)$ such that $x < y$, but $h(x) > h(y)$.
Consider $z = \lca(x,y)$. Either $h(x) > h(z)$ or $h(z) > h(y)$, contradicting
the fact that $x, y \in W_{\uparrow}\cup W_c$.
\end{proof}

\noindent By symmetry, the function $h$ restricted to the set $W_{\downarrow}\cup W_c$ is antimonotone.

A priori, points in $W_{\uparrow}$ and $W_{\downarrow}$ could be interspersed.
The next claim shows that they are in different halves of the tree $T$.

\begin{claim}\label{clm:pure}
	If $x\in W_{\downarrow}$ and $y\in W_{\uparrow}$, then $\lca(x,y)$ is the root of $T$ (which is equal to $\lceil n/2 \rceil$).
\end{claim}

\begin{proof}
	Suppose not. Let $z := \lca(x,y)$ and $w$ be the parent of $z$.
    Consider the case where $z$ is the left child of $w$,  $x$ lies in the left subtree of $z$ and $y$ lies in the right subtree of $z$.
    (All the other cases have analogous proofs.)
    Observe that all points in $Q_y$ lie in the interval $[z,w]$.
    Both $w$ and $z$ are in $Q_x$ as well as in $Q_y$.
    As $x \in W_{\uparrow}$ and $y \in W_{\downarrow}$, it must be the case that $h(w) = h(z)$.
    Since $y\notin W_{\uparrow \downarrow}$, for all $p \in Q_y$, we have $h(p) = h(w)$.
    This contradicts the fact that $y\in  W_{\uparrow}$.

    In all cases, we conclude that either $x\notin W_{\downarrow}$ or
    $y\notin W_{\uparrow}$.
    Thus, $z$ cannot have a parent, and $z = \lceil n/2 \rceil$.
\end{proof}

\begin{claim}\label{clm:obv}
Let $g:[n] \mapsto \R$ be an antimonotone function and $\dist(g, \mathrm{constant})$ denote the fraction of points that need to be changed so that $g$ is a constant function.
	If $g$ is antimonotone, and $\dist(g, \mathrm{constant}) \geq \rho$, where $\rho \le \frac 1 2$, then
	$$\Pr\limits_{u,v \in [n]: u < v} [(u,v) \textrm{ is decreasing}] \geq \frac\rho 2 .$$
\end{claim}

\begin{proof}
	The probability that $g(u) \neq g(v)$ is at least $\rho(1-\rho)$ which is at least $\frac \rho 2$ when $\rho \le \frac 1 2$. Since $g$ is antimonotone, $(u,v)$ is a decreasing pair.
\end{proof}

\noindent
Let $L$ (respectively, $R$) be the set of points in $[n] \setminus W_{\uparrow \downarrow}$ in the left (respectively, right) subtree of the root. Define $\mu_L := |L|/n$; similarly, define $\mu_R$.
Observe that both $\mu_L$ and $\mu_R$ are at least $\frac{1}{2} - \frac{\eps}{10}$.
By Claims~\ref{clm:2} and~\ref{clm:pure}, $h_{|L}$ (and $h_{|R}$) is either monotone or antimonotone.
Now, if any of these two functions were antimonotone and $\frac\eps 2$-far from being constant (w.l.o.g., assume $h_{|L}$ satisfies the condition), then by \Clm{obv}, we would have
\[
\Pr[\calE] \geq \Pr\limits_{u < v}\left[(u,v) \textrm{ is decreasing and }  u,v \in L \right] \geq  \frac\eps 4 \cdot \left(\frac{1}{2} - \frac{\eps}{10} \right)^2  \geq \frac{\eps}{25}.
\]
Assume that this doesn't occur. We have two cases.

\noindent {\bf Case 1.} Both $h_{|L}$ and $h_{|R}$ are $\frac\eps 2$-close\footnote{A function $h$ is $\eps$-close to a property $\cP$ if it is sufficient to change at most $\eps$-fraction of values in $h$ to make it satisfy $\cP$.}
to being constant. In this case, at least $(1-\frac\eps 2)|L|$ points of $L$ evaluate to a constant $C_1$, and at least $(1-\frac\eps 2)|R|$ points of $R$ evaluate to constant $C_2$. We must have $C_1 > C_2$, for otherwise, we can make $h$ monotone by changing only $\frac\eps 2 \cdot (|R|+|L|)+\frac{\eps n}{10} < \eps n$ points, which is a contradiction.
Hence,
\[
\Pr[\calE] \geq \Pr\limits_{u < v} \left[h(u) = C_1 \textrm{ and } h(v) = C_2 \right] \geq \left(1-\frac\eps 2 \right)^2 \mu_L\mu_R > \frac{1}{4} \cdot \left(\frac 1 2- \frac{\eps}{10} \right)^2 \geq \frac{\eps}{25}.
\]

\noindent {\bf Case 2.} At least one of the functions is $\frac\eps 2$-far from being constant and is monotone.
W.l.o.g., assume $h_{|L}$ satisfies this condition.
Note that all points in $L$ are only in $W_{\uparrow}\cup W_c$, and so, all points in $R$ must be in $W_{\downarrow}\cup W_c$.
This implies that $h_{|R}$ is antimonotone. (Note that a constant function is also antimonotone.)
But then, $h_{|R}$ must be $\frac \eps 2$-close to being constant.
Then at least $(1-\frac\eps 2)|R|$ points in $R$ evaluate to a constant, say $C$.
Let $U$ denote the set of points in $L$ whose values are strictly
greater than $C$. Since $h_{|L}$ is monotone, we can make
$h$ monotone by deleting all points in $U, W_{\uparrow \downarrow}$,
and the points in $R$ that do not evaluate to $C$.
The total number of points to be deleted is at most $|U| + \frac{\eps n}{10} + \frac{\eps n}{2}$, which must be at least $\eps n$, as $h$ is $\eps$-far from monotone. Hence, $|U| > \eps n/3$.
Therefore,
\[
\Pr[\calE] \geq \Pr\limits_{u < v} \left[ u \in U \textrm{ and } h(v) = C \right] \geq \frac{\eps}{3} \cdot  \left(1 - \frac\eps 2 \right)\mu_R  > \frac{\eps}{25} .
\]
This completes the proof of \Lem{treetester}.
\end{proof}

We now analyze \Alg{na-unate-hg}.
It is evident that it has one-sided error and makes
$O(\frac d \eps(\log n + \log\frac d\eps))$ queries.
It suffices to prove the following.

\begin{theorem} \label{thm:non-adap-hyper} If $f:[n]^d \mapsto \R$ is $\eps$-far
	from unate, then \Alg{na-unate-hg} rejects with probability at least $2/3$.
\end{theorem}

\begin{proof} For any line $\ell$, we define the following quantities.
\begin{compactitem}
\item $\alpha_\ell$: the distance of $f_{|\ell}$ to monotonicity.
\item $\beta_\ell$: the distance of $f_{|\ell}$ to antimonotonicity.
\item $\sigma_\ell$: the probability that \Alg{tree-tester}, on input $f_{|\ell}$, returns $\{\uparrow, \downarrow\}$.
\item $\delta_\ell$: the probability that a uniformly random pair in $\ell$ is decreasing.
\item $\lambda_\ell$: the probability that a uniformly random pair in $\ell$ is increasing.
\end{compactitem}

\noindent Let $L_i$ be the set of $i$-lines. By 
\Thm{dimred},
	$$\frac{1}{n^{d-1}} \sum_{i=1}^d \min\left(\sum_{\ell \in L_i} \alpha_\ell, \sum_{\ell \in L_i} \beta_\ell \right) \geq \frac{\eps}{4}.$$
	By \Lem{treetester}, for every line $\ell$, we have
	$\sigma_\ell + \delta_\ell \geq \alpha_\ell/25$ and $\sigma_\ell + \lambda_\ell \geq \beta_\ell/25$.
	Also note,
	$$
	\frac{1}{n^{d-1}}\sum_{i=1}^d \left[\sum_{\ell \in L_i} \sigma_\ell + \min\left(\sum_{\ell \in L_i} \delta_\ell, \sum_{\ell \in L_i} \lambda_\ell \right)\right] \geq \frac{1}{n^{d-1}}\sum_{i=1}^d  \min\left(\sum_{\ell \in L_i} (\sigma_\ell + \delta_\ell), \sum_{\ell \in L_i} (\sigma_\ell + \lambda_\ell) \right)$$
	Combining these bounds, we obtain that the LHS is at least $\eps/100$.
	%
	%\[
	%\frac{1}{n^{d-1}} \sum_{i=1}^d \left[\sum_{\ell \in L_i} \sigma_\ell + \min\left(\sum_{\ell \in L_i} \delta_\ell, \sum_{\ell \in L_i} \lambda_\ell \right)\right]  \geq  \frac{1}{n^{(d-1)}} \sum_{i=1}^d \min\left(\sum_{\ell \in L_i} (\sigma_\ell + \delta_\ell), \sum_{\ell \in L_i} (\sigma_\ell + \lambda_\ell) \right)\]
	%
	Note that the first term,
	which is equal to $\sum_{i=1}^d \EX_{\ell \in L_i} [\sigma_\ell]$,
	is the expected
	number of times a single iteration of Steps~\ref{step:mono-start}-\ref{step:log}
	rejects. If this quantity is at least $\eps/200$, then
	the tester rejects with probability at least $2/3$. If not,
	%then $n^{-(d-1)} \sum_i \sum_{\ell \in L_i} \min(\delta_\ell, \lambda_\ell) \geq \eps/200$.
	then we have $n^{-(d-1)} \sum_{i=1}^d \min(\sum_{\ell \in L_i} \delta_\ell, \sum_{\ell \in L_i} \lambda_\ell) \geq \eps/200$.
	Using a calculation identical to that of the proof of Lemma~\ref{lem:na-unate-rej},
	the probability that Step~\ref{step:rej-hg1} rejects in some iteration is at least
	$2/3$.
\end{proof}


\iffalse
\noindent
It is evident that \Alg{na-unate-hg} has one-sided error and makes
$O(\frac d \eps(\log n + \log\frac d\eps))$ queries.
It suffices to prove the following.

\begin{theorem} \label{thm:non-adap-hyper} If $f:[n]^d \mapsto \R$ is $\eps$-far
from unate, then \Alg{na-unate-hg} rejects with probability at least $2/3$.
\end{theorem}

\begin{proof} For any line $\ell$, we define the following quantities.
\begin{asparaitem}
    \item $\alpha_\ell$: the distance of $f_{|\ell}$ to monotonicity.
    \item $\beta_\ell$: the distance of $f_{|\ell}$ to antimonotonicity.
    \item $\sigma_\ell$: the probability that \Alg{tree-tester}, on input $f_{|\ell}$, returns $\{\uparrow, \downarrow\}$.
    \item $\delta_\ell$: the probability that a uniformly random pair in $\ell$
    is decreasing.
    \item $\lambda_\ell$: the probability that a uniformly random pair in $\ell$ is increasing.
\end{asparaitem}
\noindent
Let $L_i$ be the set of $i$-lines. By %the dimension reduction of
\Thm{dimred},
$$\frac{1}{n^{d-1}} \sum_{i=1}^d \min\left(\sum_{\ell \in L_i} \alpha_\ell, \sum_{\ell \in L_i} \beta_\ell \right) \geq \eps/4.$$
By \Lem{treetester}, for every line $\ell$, we have
$\sigma_\ell + \delta_\ell \geq \alpha_\ell/25$ and $\sigma_\ell + \lambda_\ell \geq \beta_\ell/25$.
Also note,
$$
\frac{1}{n^{d-1}}\sum_{i=1}^d \left[\sum_{\ell \in L_i} \sigma_\ell + \min\left(\sum_{\ell \in L_i} \delta_\ell, \sum_{\ell \in L_i} \lambda_\ell \right)\right] \geq \frac{1}{n^{d-1}}\sum_{i=1}^d  \min\left(\sum_{\ell \in L_i} (\sigma_\ell + \delta_\ell), \sum_{\ell \in L_i} (\sigma_\ell + \lambda_\ell) \right)$$
Combining these bounds, we obtain that the LHS is $\geq \eps/100$.
%
%\[
%\frac{1}{n^{d-1}} \sum_{i=1}^d \left[\sum_{\ell \in L_i} \sigma_\ell + \min\left(\sum_{\ell \in L_i} \delta_\ell, \sum_{\ell \in L_i} \lambda_\ell \right)\right]  \geq  \frac{1}{n^{(d-1)}} \sum_{i=1}^d \min\left(\sum_{\ell \in L_i} (\sigma_\ell + \delta_\ell), \sum_{\ell \in L_i} (\sigma_\ell + \lambda_\ell) \right)\]
%
Note that the first term,
which is equal to $\sum_{i=1}^d \EX_{\ell \in L_i} [\sigma_\ell]$,
is the expected
number of times a single iteration of Steps~\ref{step:mono-start}-\ref{step:log}
rejects. If this quantity is at least $\eps/200$, then
the tester rejects with probability at least $2/3$. If not,
%then $n^{-(d-1)} \sum_i \sum_{\ell \in L_i} \min(\delta_\ell, \lambda_\ell) \geq \eps/200$.
then we have $n^{-(d-1)} \sum_{i=1}^d \min(\sum_{\ell \in L_i} \delta_\ell, \sum_{\ell \in L_i} \lambda_\ell) \geq \eps/200$.
Using a calculation identical to that of the proof of Lemma~\ref{lem:na-unate-rej},
the probability that Step~\ref{step:rej-hg1} rejects in some iteration is at least
$2/3$.

\end{proof}
%\end{proof}
\fi

\chapter{Lower Bounds}

\section{The Lower Bound for Nonadaptive Testers over Hypercubes} \label{sec:lb}
In this section, we prove \Thm{non-adap-lb-1}, which gives a lower bound for nonadaptive unateness testers for functions over the hypercube.

Fischer~\cite{Fis04} showed that in order to prove lower bounds for a general class of properties on the line domain, it is sufficient to consider a special class of testers called {\em comparison-based testers}. The properties he looked at are called {\em order-based properties} (see \Def{obp}) and they include monotonicity and unateness.
A tester is {\em comparison-based} if it bases its decisions only on the {\em order} of the function values at the points it queried,  and not on the values themselves.
Chakrabarty and Seshadhri~\cite{CS14} extended Fischer's proof to monotonicity on any partially-ordered domain. As we show in \Sec{reduction-cbt} below, Chakrabarty and Seshadhri's proof goes through for all order-based properties on partially-ordered domains. We include this proof for completeness, filling in the details needed to generalize the original proof.

Our main technical contribution is the construction of a distribution of functions $f:\{0,1\}^d \to \R$ on which every nonadaptive comparison-based tester must query $\Omega(d \log d)$ points to determine whether the sampled function is unate or far from unate.
We describe this construction in \Sec{hard-dist} and show its correctness in Sections~\ref{sec:f-graph}-\ref{sec:bad}.

%New section on COmparison based testers reduction
\subsection{Reduction to Comparison-Based Testers}\label{sec:reduction-cbt}
In this section, we prove that if there exists an $\eps$-tester for an order-based property of functions over a partially-ordered domain, then there exists a comparison-based $\eps$-tester for the same property making the same number of queries. This is stated in \Thm{CS14}.
Before stating the theorem, we introduce several definitions.

\begin{definition} \label{def:t-eps-del-tester}
A $(t,\eps,\delta)$-tester for a property is a ($2$-sided error) $\eps$-tester making at most $t$ queries, that errs with probability at most $\delta$.
\end{definition}

\begin{definition}[Order-based property]\label{def:obp}
For an arbitrary partial order $D$ and an arbitrary total order $R$, a property $\cP$ of functions $f:D \to R$ is {\em order-based} if,
for all strictly increasing maps $\phi: R \to R$ and all functions $f$, we have $\dist(f,\cP) = \dist(\phi\circ f, \cP)$.
\end{definition}

\noindent Specifically, unateness is an order-based property.
The following theorem is an extension of Theorem~5 in~\cite{Fis04} and Theorem~2.1 in~\cite{CS14}.
In particular, Theorem~2.1 in~\cite{CS14} was proved with the assumption that the function values are distinct. We generalize 
%Theorem~2.1 in~\cite{CS14}
the theorem
by removing this assumption.

\begin{theorem}[implicit in~\cite{Fis04,CS14}]\label{thm:CS14}
Let $\cP$ be an order-based property of functions $f:D \to \N$. Suppose there exists a $(t,\eps,\delta)$-tester for $\cP$.
Then there exists a comparison-based $(t,\eps,2\delta)$-tester
for $\cP$.
\end{theorem}

The rest of this section is devoted to proving \Thm{CS14}.
Our proof closely follows the proof of Theorem~2.1 in~\cite{CS14}.
The proof has two parts. In the first part, we describe a reduction from a tester to a {\em discretized tester} and, in the second part, we describe a reduction from a discretized tester to a comparison-based tester.

Let $\cP$ be a property of functions $f:D \to R$ for an arbitrary partial order $D$ and an arbitrary total order $R \subseteq \N$. Let $\cT$ be a $(t,\eps,\delta)$-tester for $\cP$. First, we define a family of probability functions that completely characterizes $\cT$. Fix some $s \in [t]$. Consider the point in time in an execution of the tester $\cT$ on some input function $f$, where exactly $s$ queries have been made. 
Suppose these queries are $x_1, x_2, \ldots, x_s \in D$ and the corresponding answers are $a_1=f(x_1), a_2=f(x_2), \ldots, a_s=f(x_s)$. 
Let {\em query vector} $X$ be $(x_1, \ldots, x_s)$ and {\em answer vector} $A$ be $(a_1, \ldots, a_s)$. The next action of the algorithm is either choosing the $\ord{(s+1)}$ query from $D$ or outputting {\em accept} or {\em reject}. For each action $y \in D \cup \{\texttt{accept}, \texttt{reject}\}$, let $p_X^y(A)$ denote the probability that $\cT$ chooses action $y$ after making queries $X$ and receiving answers $A$. Since $p_X^y(A)$ is a probability distribution,
$$\forall s < t, \forall X \in D^s, \forall A \in R^s \sum_{y \in D \cup \{\texttt{accept}, \texttt{reject} \}} p_X^y(A) = 1.$$
Furthermore, the tester cannot make more than $t$ queries,
and so the action $(t+1)$ must be either \texttt{accept} or \texttt{reject}. Formally,
$$\forall X \in D^t, \forall A \in R^t \sum_{y \in \{\texttt{accept}, \texttt{reject} \}} p_X^y(A) = 1.$$

\begin{definition} [Discretized tester] \label{def:disc}
A tester $\cT$ is {\em discretized} if all $p_X^y(A)$-values associated with $\cT$ come from the range $\left\{ \frac{i}{K}: i\in \{0,1,\ldots,K\} \right\}$ for some integer $K$.
\end{definition}

Chakrabarty and Seshadhri~\cite{CS14} proved that if there exists a $(t,\eps,\delta)$-monotonicity tester $\cT$ for functions $f:D \to \N$, then there exists a discretized $(t,\eps,2\delta)$-monotonicity tester $\cT'$ for the same class of functions.
Both the statement and the proof in~\cite{CS14} hold not only for testers of monotonicity, but for testers of all properties of functions $f:D \to R$.

\begin{lemma}[implicit in~{\cite[Lemma~2.2]{CS14}}] \label{lem:disc-test}
Suppose there exists a $(t,\eps,\delta)$-tester $\cT$ for a property $\cP$ of functions $f:D \to R$.
Then, there exists a $(t,\eps,2\delta)$-discretized tester $\cT'$ for $\cP$.
\end{lemma}

\noindent This completes the first part of the proof.

Next, we will show how to transform a discretized tester into a comparison-based tester.
Intuitively, a tester is comparison-based if each query of the tester depends only on the ordering of the answers to the previous queries, not on the values themselves.
We define a family of probability functions $q$ in order to characterize comparison-based testers.
The $q$-functions are defined in terms of $p$-functions, but, in their definition, we decouple the set of values that were received as answers from their positions in the answer vector.
Let $V$ represent the set $\{a_1, \ldots, a_s\}$ of answer values (without duplicates).
Let $r$ be the number of (distinct) values in $V$. Note that $r \leq s$.
Suppose, $V$ is $\{v_1, v_2, \ldots, v_r\}$ where $v_1, \ldots, v_r \in R$ and $v_1 < v_2 < \ldots < v_r$.
Let $\rho$ be the map from positions of values in the answer vector to their corresponding indices in $V$, that is, $\rho:[s] \to [r]$. Observe that $\rho$ is surjective.
The $q$-functions are defined as follows:
$$q_{X,\rho}^y(V) = p_X^y((v_{\rho(1)}, v_{\rho(2)}, \ldots,v_{\rho(s)})).$$
%Similar to $p$-functions, the $q$-functions also have as arguments, a query vector $X$ and a next action $y$.
%Instead of receiving the vector containing answers $A$, the $q$ functions receive a set $V$ and a surjection $\rho$ that dictates the ordering of the answers received.
%This is where the proof differs from that of~\cite{CS14}; their proof is based on the assumption that the function values are distinct, hence, they define a permutation $\sigma$ in place of the surjection $\rho$.
%We generalize the theorem by removing this assumption.
%The function $q$ receives a subset of queries, sorts them in increasing order, constructs a new larger vector according to a surjective function and passes the vector to the function $p$. 
%\new{
%Formally we define $q$ as follows. Let $R^{(r)}$ denote 
%the set of all unordered subsets of $R$ of size $r$. Let $X \in D^s$ be a query vector of size $s \geq r$, function $\rho:[s] \to [r]$ be surjective and $y \in D \cup \{\texttt{accept}, \texttt{reject}\}$ be an action. We define $q_{X,\rho}^y: R^{(r)} \to [0,1]$ with the semantics,
%$$ \text{for any set } V = (v_1 < v_2 < \ldots < v_r) \in R^{(r)}, \ \ \  $$}
%\new{Now, we formally define comparison-based testers.}
%We introduce a new family of functions. For all vectors $X \in D^s$, queries $y \in D \cup \{\texttt{acc}, \texttt{rej}\}$, sets $V \in R^{(r)}$ and all surjective functions $\rho:[s] \to [r]$ such that $r \leq s$, define $q_{X,\rho}^y: R^{(r)} \to [0,1]$ with the semantics,
%$$ \text{for any set } V = (v_1 < v_2 < \ldots < v_r) \in R^{(r)}, \ \ \  q_{X,\rho}^y(V) = p_X^y((v_{\rho(1)}, v_{\rho(2)}, \ldots,v_{\rho(s)})).$$
%In other words, $q_{X,\rho}^y$ receives a subset of $R$ of size $r$, sorts them in increasing order, constructs a vector of size $s$ according to the surjection $\rho$ and passes the vector to $p_X^y$.
%Since $\rho$ is surjective, all elements in $V$ appear at least once in the vector.
%Now, using these $q$ functions, we can define comparison-based testers.

\noindent Let $R^{(r)}$ denote the set of all subsets of $R$ of size $r$.

\begin{definition} [Comparison-based tester]\label{def:cbt}
A tester $\cT$ for an order-based property $\cP$ is {\em comparison-based} for functions $f:D \to R$, if for all $r,s$ satisfying $ r \leq s \leq t$, and all $X \in D^s$, $y \in D \cup \{\texttt{accept}, \texttt{reject}\}$ and surjections $\rho:[s] \to [r]$, the function $q_{X,\rho}^y$ is constant on $R^{(r)}$. That is, for all $V, V' \in R^{(r)}$, we have $q_{X,\rho}^y(V) = q_{X,\rho}^y(V')$.
\end{definition}

%\noindent In other words, a tester is comparison-based if the $\ord{(s+1)}$ query of the tester depends only on the ordering of the answers of the first $s$ queries, not on the values themselves.

%\new{
%\begin{lemma}\label{lem:disc-to-comp}
%For any discretized tester $\cT'$ for an order-based property $\cP$ over the functions $f:D \to \N$, there exists an infinite set $R \subseteq \N$ such that, on functions $f:D \to R$, the tester $\cT'$ behaves as a comparison-based tester.
%\end{lemma}}
%
%\begin{proof}
%\new{
%The proof of existence of the infinite set $ R \subseteq \N$ is done through Ramsey theory arguments. So first we introduce some Ramsey theory terminology. For any positive integer $i$, a {\em finite coloring} of $\N^{(i)}$ is a function $\texttt{col}_i:\N^{(i)} \to [C]$ for a finite $C$, where $[C]$ represents a set of colors. An infinite set $Z \subseteq \N$ is {\em monochromatic} with respect to $\texttt{col}_i$ if for all $i$-sized subsets $U, V \in Z^{(i)}$ we have $\texttt{col}_i(U) = \texttt{col}_i(V)$. A $k$-wise finite coloring of $\N$, for a positive integer $k$, is a collection of $k$-colorings $\texttt{col}_1, \texttt{col}_2, \ldots, \texttt{col}_k$, where each coloring is over subsets of different sizes.
%An infinite subset $Z \subseteq \N$ is $k$-wise monochromatic if $Z$ is monochromatic with respect to all $\texttt{col}_i$'s, such that $i$ is a positive integer.}
%\end{proof}

\noindent To complete the proof of \Thm{CS14}, we show that if there exists a discretized tester $\cT$ for an order-based property $\cP$ over the functions $f:D \to \N$, then there exists an infinite set $R \subseteq \N$ such that, for functions $f:D \to R$, the tester $\cT$ is comparison-based.
The existence of this infinite set $R$ is proved using Ramsey theory arguments.

We introduce some Ramsey theory terminology. 
Consider an integer $C$, where $[C]$ represents a set of colors.
For any positive integer $i$, a {\em finite coloring} of $\N^{(i)}$ is a function $\texttt{col}_i:\N^{(i)} \to [C]$.
An infinite set $R \subseteq \N$ is {\em monochromatic} with respect to $\texttt{col}_i$ if for all $i$-sized subsets $V, V' \in R^{(i)}$, the color $\texttt{col}_i(V) = \texttt{col}_i(V')$. A {\em $k$-wise finite coloring of $\N$} is a collection of $k$-colorings $\texttt{col}_1, \texttt{col}_2, \ldots, \texttt{col}_k$.
Note that each coloring $\texttt{col}_1, \ldots, \texttt{col}_k$ is defined over subsets of different sizes.
An infinite subset $R \subseteq \N$ is {\em $k$-wise monochromatic} \new{with respect to $col_1, \ldots, col_k$} if $R$ is monochromatic with respect to all $\texttt{col}_i$ for $i \in [k]$.

We use the following variant of Ramsey's theorem which was also used in~\cite{Fis04,CS14}.

\begin{theorem}[Theorem~2.3 in~\cite{CS14}]\label{thm:ramsey}
For any $k$-wise finite coloring of $\N$, there exists an infinite $k$-wise monochromatic subset $R \subseteq \N$.
\end{theorem}


\begin{proof}[Proof of \Thm{CS14}]
Suppose there exists a $(t,\eps,\delta)$-tester for property $\cP$ of functions $f:D \to \N$. By \Lem{disc-test}, there exists a $(t,\eps,2\delta)$-discretized tester $\cT$ for $\cP$. \new{Special name for family of $q$ functions?} Let 
$q$ %$q_{X, \rho}^y$ 
be the family of probability functions that characterizes $\cT$.

We define a $t$-wise finite coloring of $\N$. For each $r \in [t]$ and $V \in \N^{(r)}$, the color $\texttt{col}_r(V)$ is defined as a vector of probability values $q_{X,\rho}^y(V)$. 
\new{The vector is indexed by $(y, X, \rho)$ for each $y \in D \cup \{\texttt{accept}, \texttt{reject}\}, s$ satisfying $r \leq s \leq t$ and $X \in D^s$ and surjection $\rho:[r] \to [s]$.}
The value at the index $(y,X,\rho)$ in $\texttt{col}_r(V)$ is equal to $q_{X, \rho}^y (V)$.
Note that, \new{there are finitely many possible values for $y$ and $X$, and surjections $\rho$.}
%Note that, for each $s$ satisfying $r \leq s \leq t$,
%the number of different values that $y$ takes, the size of $D^s$ and the number of possible surjections $\rho$ are all finite.
So, the dimension of the vector $\texttt{col}_r(V)$ is finite. Furthermore, since the tester is discretized, the number of different values that the $q$-functions take is also finite. Hence, the range of $\texttt{col}_r$ is finite.
Now, we have a $t$-wise finite coloring $\texttt{col}_1, \ldots, \texttt{col}_t$ of $\N$. 
By \Thm{ramsey}, there exists an infinite $t$-wise monochromatic set $R \subseteq \N$.
Thus, for each $r \in [t]$ and $V, V' \in R^{(r)}$, we have $\texttt{col}_r(V) = \texttt{col}_r(V')$, implying that $q_{X,\rho}^y (V) = q_{X,\rho}^y (V')$ for all $y, X, \rho$. Thus, $\cT$ is comparison-based for functions $f:D \to R$.

Consider a strictly monotone increasing map $\phi: \N \to R$. Given any function $f: D \to \N$, consider $\phi \circ f: D \to R$. 
Define an algorithm $\cT'$, which on input $f$, runs $\cT$ on $\phi \circ f$.
Since $\cP$ is order-based, $\text{dist}(f, \cP) = \text{dist}(\phi \circ f, \cP)$. 
Hence, $\cT'$ is a $(t,\eps, 2\delta)$-tester for $\cP$.
%As $\cT'$ runs $\cT$ on $\phi \circ f:D \to R$ , and $\cT$ is comparison based on the range $R$.
%Hence, $\cT'$ is a comparison-based tester.
\new{Moreover, since the tester $T'$ just runs $T$ on a input $\phi \circ f: D \to R$ as a subroutine and $\cT$ is comparison-based for that input, the tester $\cT'$ is also comparison-based.}
\end{proof}


\subsection{The Hard Distributions} \label{sec:hard-dist}
Our main lower bound theorem is stated next. \new{Together with \Thm{CS14}}, it implies Theorem~\ref{thm:non-adap-lb-1}.

\begin{theorem}\label{thm:non-adap-lb}
Any nonadaptive comparison-based unateness tester of functions $f:\{0,1\}^d\to \R$ must make $\Omega(d\log d)$ queries.
\end{theorem}

\noindent \new{The proof of \Thm{non-adap-lb} is presented in Sections~\ref{sec:hard-dist}-\ref{sec:bad} and forms the core technical content of this work.}

By \Thm{CS14} and Yao's minimax principle~\cite{Yao77}, it suffices to prove
the lower bound for deterministic, nonadaptive, comparison-based testers over a known distribution of functions.
It may be useful for the reader to recall the  sketch of the main ideas given in Section~\ref{sec:intro-tech}.
For convenience, assume $d$ is a power of $2$ and let $d' := d+\log_2d$.
We will focus on functions $h:\{0,1\}^{d'} \to \R$,
and prove the lower bound of $\Omega(d \log d)$ for this class of functions,
as $\Omega(d \log d) = \Omega(d' \log d')$.


We partition $\{0,1\}^\dd$ into $d$ subcubes based on the most significant $\log_2 d$ bits.
Specifically, for $i \in [d]$, the $\ord{i}$ subcube is defined as
\[C_i := \{x\in \{0,1\}^\dd: \dec(x_{d'}x_{d'-1}\cdots x_{d+1}) = i - 1\},\]
where $\dec(z) := \sum_{i = 1}^p z_i 2^{i-1}$ denotes the integer equivalent of the binary string $z_p z_{p-1} \ldots z_1$.

Let $m=d$. We denote the set of indices of the subcube by $[m]$ and the set of dimensions by $[d]$.
We use $i,j\in [m]$ to index subcubes,
and $a,b\in [d]$ to index dimensions.
We now define a series of random variables, where each subsequent variable may depend on the previous ones.
\begin{compactitem}
    \item $k$: a number picked uniformly at random from $\left[\frac{1}{2}\log_2 d \right]$.
    \item $R$: a uniformly random subset of $[d]$ of size $2^k$.
    \item $r_i$: for each $i \in [m]$, $r_i$ is picked from $R$ uniformly and independently at random.
    \item $\alpha_b$: for each $b \in [d]$, $\alpha_b$ is picked from $\{-1,+1\}$ uniformly and independently at random. (Note: $\alpha_b$ only needs to be defined for each $b \in R$. We define it over $[d]$ just so that it is independent of $R$.)
    \item $\beta_i$: for each $i \in [m]$, $\beta_i$ is picked from $\{-1,+1\}$ uniformly and independently at random.
\end{compactitem}

\noindent We denote the tuple $(k,R,\{r_i\})$ by $\bS$, also referred
to as the \emph{shared randomness}. We use $\bT$ to refer
to the entire set of random variables $(k, R, \{r_i\}, \{\alpha_b\}, \{\beta_i\})$.
\new{
Given $\bT$, define the functions
\begin{align*}
f_{\bT}(x) &:=  \sum_{b \in [d'] \setminus R} x_b 3^b + \alpha_{r_i} x_{r_i}3^{r_i}, %\textrm{ where $i$ is the subcube with } x \in C_i.
\\
g_{\bT}(x) &:=  \sum_{b \in [d'] \setminus R} x_b 3^b + \beta_i x_{r_i}3^{r_i} %\textrm{ where $i$ is the subcube with } x \in C_i.
\end{align*}
where $i$ is the subcube containing $x$, i.e., $i = \dec(x_{d'}x_{d'-1}\cdots x_{d+1}) + 1$.
The distributions $\Yes$ and $\No$ generate $f_{\bT}$ and $g_{\bT}$, respectively.
}

In all cases, the function restricted to any subcube $C_i$ is linear.
Consider some dimension $b \in R$. 
\new{
There can be several $i \in [m]$ such that $r_i = b$.}
%$r_i$'s that are equal to $b$. 
For $f_{\bT}$, in all of these subcubes,
the coefficient of $x_{r_i}$ has the same sign, namely $\alpha_{r_i}$. % and this makes all these functions unate.
For $g_{\bT}$, the coefficient $\beta_i$ is potentially different,
as it depends on the actual subcube.

\noindent We write $f \sim \cD$ to denote that $f$ is sampled from distribution $\cD$.

\begin{claim}\label{clm:yes}
Every function $f \sim \Yes$ is unate.
\end{claim}

\begin{proof}
\new{Fix some $f \in \supp(\Yes)$. Since $f_{|C_i}$ is linear,
it suffices to argue that, for any $b \in [d']$, the coefficient of $x_b$ (when it is non-zero) has the same sign in all subcubes.
When $b \in [d'] \setminus R$, the coefficient of $x_b$ is always $3^b$. 
If $b \in R$, then the coefficient is either $0$ or $3^b\alpha_b$.}
\end{proof}

\begin{claim}\label{clm:no}
A function $g \sim \No$ is $\frac{1}{8}$-far from unate with probability at least $ 9/10$.
\end{claim}

\begin{proof}
	Note that $|R| \leq \sqrt{d}$.
	For any $r \in R$, let $A_r := \{ i : r_i = r \}$, the set of subcube indices with $r_i = r$.	
	Observe that $\EX[|A_r|] \geq m/\sqrt{d} = \sqrt{d}$. By Chernoff bound and union bound, for all $r \in R$,
	we have $|A_r| \geq \sqrt{d}/2$	with probability at least $1 - d\exp(-\sqrt{d}/8)$.
	
	Condition on the event that $|A_r| \geq \sqrt{d}/2$ for all $r \in R$.
	For each $i \in A_r$, there is a random choice of $\beta_i$.
	Partition $A_r$ into $A^+_r$ and $A^-_r$,
	depending on whether $\beta_i$ is $+1$ or $-1$, respectively. Again, by a Chernoff bound
	and union bound, for all $r \in R$, we have $\min(|A^+_r|,|A^-_r|) \geq |A_r|/4$ with probability at least $1-d\exp(-\sqrt{d}/32)$.
	\new{Thus, we can assume that the event $\min(|A^+_r|,|A^-_r|) \geq |A_r|/4$ holds with probability at least $1-d(\exp(-\sqrt{d}/8)+\exp(-\sqrt{d}/32))$, which is at least $9/10$, for large enough $d$ and}
	for any choice of $k$ and $R$. 
	
	Denote the size of any subcube $C_i$ by $s$.
	In $g_{\bT}$, for all $i \in A^+_r$, all $r$-edges in $C_i$ are increasing, whereas, for all $j \in A^-_r$, all $r$-edges in $C_j$ are decreasing. 
\new{To make $g_{\bT}$ unate, all these edges must have the same direction (i.e., increasing or decreasing).}
	This requires modifying at least $\frac{s}{2} \cdot \min(|A^+_r|,|A^-_r|) \geq \frac{s|A_r|}{8}$ values in $g_{\bT}$. Summing over all $r$, we need to change at least $\frac{s}{8}\sum_r |A_r|$
	values. Since the $A_r$'s partition the set of subcubes, this corresponds to at least a $\frac{1}{8}$-fraction of the domain.
\end{proof}


\subsection{From Functions to Signed Graphs that are Hard to Distinguish} \label{sec:f-graph}
For convenience, denote $x \prec y$ if $\dec(x) < \dec(y)$. Note that $\prec$ forms a total ordering
on $\{0,1\}^{\dd}$.
Given $x \prec y\in \{0,1\}^\dd$ and a function $h:\{0,1\}^\dd\to \R$, define
$\sgn_h(x,y)$ to be $1$ if $h(x) < h(y)$, $0$ if $h(x) = h(y)$,
and $-1$ if $h(x) > h(y)$.

Any deterministic, nonadaptive, comparison-based tester is defined as follows:
It makes a set of queries $Q$ and decides whether or not the input function $h$ is unate
depending on the $|Q|\choose{2}$-comparisons in $Q$.
More precisely, for every pair $(x,y) \in Q \times Q$,
$x \prec y$, we insert
an edge labeled with $\sgn_h(x,y)$. Let this signed graph be called $G^Q_h$.
Any nonadaptive, comparison-based algorithm can be described
\new{as a method to partition the universe of all signed graphs over $Q$  into $\cG_Y$ and $\cG_N$.}
%as a partition of the universe of all signed graphs over $Q$  into $\cG_Y$ and $\cG_N$.
The algorithm accepts the function $h$ iff $G^Q_h \in \cG_Y$.

Let $\bG^Q_Y$ be the distribution of the signed graphs $G^Q_h$ when $h\sim \Yes$. Similarly, define $\bG^Q_N$ when $h\sim \No$. Our main technical theorem is \Thm{tv}, which is proved in \Sec{thm-tv-proof}.
\begin{theorem}\label{thm:tv}
For small enough $\delta > 0$ and large
enough $d$, if $|Q| \leq \delta d\log d$, then $\|\bG^Q_Y - \bG^Q_N\|_{\mathrm{TV}} = O(\delta)$.
\end{theorem}

\noindent We now prove that \Thm{tv} implies \Thm{non-adap-lb}, the main lower bound.

\begin{proof}[{\bf Proof of Theorem~\ref{thm:non-adap-lb}}]
	Consider the distribution over functions where with probability $1/2$, we sample from $\Yes$ and with the remaining probability we sample from $\No$.
	By \Thm{CS14} and Yao's minimax principle,
it suffices to prove that any deterministic, nonadaptive, comparison-based tester making
at most $\delta d\log d$ queries (for small enough $\delta > 0$) errs with probability at least $ 1/3$. Now, note that
%
\begin{align*}
\Pr[\textrm{error}] = \frac{1}{2} \cdot \Pr_{h\sim \Yes} [G^Q_h \in \cG_N] 
	+ \frac{1}{2} \cdot \Pr_{h\sim \No} [G^Q_h \in \cG_Y \text{ and } h \ \textrm{is $\frac{1}{8}$-far from unate}].
\end{align*}

\noindent By \Thm{tv}, the first term is at least $\frac{1}{2}\cdot \left(\Pr_{h\sim \No} [G^Q_h \in \cG_N]  - O(\delta) \right)$, and by \Clm{no},
	the second term is at least $\frac{1}{2}\cdot \left(\Pr_{h\sim \Yes} [G^Q_h \in \cG_Y] -O(\delta) - \frac{1}{10}\right)$. Summing them up, we get
$\Pr[\textrm{error}] \geq \frac{1}{2} - O(\delta) - \frac{1}{20}$ which is at least $\frac 1 3$ for small enough $\delta$.
\end{proof}

The proof of \Thm{tv} is naturally tied to the behavior of $\sgn_h$.
Ideally, we would like to say that $\sgn_h(x,y)$ is almost identical
regardless of whether $h \sim \Yes$ or $h \sim \No$. Towards this,
we determine exactly the set of pairs $(x,y)$ that potentially
differentiate $\Yes$ and $\No$.

\begin{claim}\label{clm:inv}
\new{For all $h \in \supp(\Yes) \cup \supp(\No), x \in C_i$ and $y \in C_j$ such that $i < j$, we have $\sgn_h(x,y) = 1$.}
%For all $h \in \supp(\Yes) \cup \supp(\No)$, for all $x \in C_i$ and $y \in C_j$ such that $i < j$, we have $\sgn_h(x,y) = 1$.
\end{claim}

\begin{proof}
	For any $h$, we can write $h(x)$ as $\sum_{b > d} 3^b \cdot x_b + \sum_{b \leq d} c_b(x) \cdot 3^b \cdot x_b$,
	where $c_b: \{0,1\}^{d'} \to \{-1,0,+1\}$. Thus,
	$h(y) - h(x) = \sum_{b > d} 3^b(y_b - x_b) + \sum_{b \leq d} 3^b(c_b(y) \cdot y_b - c_b(x) \cdot x_b)$.
	Recall that $x \in C_i , y \in C_j$, and $j > i$. Let $q$ denote
	the most significant bit of difference between $x$ and $y$. We have
	$q > d$, and $y_q = 1$ and $x_q = 0$. Note that for $b \leq d$, $|c_b(y) \cdot y_b - c_b(x) \cdot x_b)| \leq 2$.
	Thus, $h(y) - h(x) \geq 3^q - 2\sum_{b < q} 3^b > 0$.
\end{proof}

\noindent Thus, comparisons between points in different subcubes reveal no information
about which distribution $h$ was generated from. Therefore, the ``interesting'' pairs that can distinguish whether $h \sim \Yes$ or $h \sim \No$ must lie in the same subcube.
The next claim shows a further criterion that is needed for a pair to be interesting.
We first define another notation \new{needed for the claim}.

\begin{definition} \label{def:t} 
For any setting of the shared randomness $\bS$,
subcube $C_i$, and points $x,y \in C_i$, we define $\coord{i}{\bS}(x,y)$
to be the most significant coordinate of difference (between $x,y$)
in $([d] \setminus R) \cup \{r_i\}$.
%that is not in $R\setminus r_i$.
\end{definition}

\noindent Note that $\bS$ determines $R$ and $\{r_i\}$.
\new{For any $\bT$ that extends $\bS$, the restriction of both $f_{\bT}$ and $g_{\bT}$ to $C_i$ is unaffected by coordinates in $R \setminus \{r_i\}$.}
%For any $\bT$ that extends $\bS$ and any function,
%the restriction to $C_i$ is unaffected by the coordinates in $R \setminus r_i$.
Thus, $\coord{i}{\bS}(x,y)$ is the first coordinate of difference that is influential
in $C_i$.

\begin{claim}\label{clm:interesting}
Fix some $\bS$, subcube $C_i$, and points $x,y \in C_i$.
Let $c = \coord{i}{\bS}(x,y)$, and assume $x \prec y$.
For any $\bT$ that extends $\bS$:
\begin{compactitem}
    \item If $c \neq r_i$, then $\sgn_{f_{\bT}}(x,y) = \sgn_{g_{\bT}}(x,y) = 1$.
    \item If $c = r_i$, $\sgn_{f_{\bT}}(x,y) = \alpha_{c}$ and $\sgn_{g_{\bT}}(x,y) = \beta_i$.
\end{compactitem}
\end{claim}

\begin{proof}
Assume $x \in C_i$.
	Recall that $f_{\bT}(x) = \sum_{b \in [d'] \setminus R} x_b 3^b + \alpha_{r_i}\cdot x_{r_i} 3^{r_i}$ and
	$g_{\bT}(x) = \sum_{b \in [d'] \setminus R} x_b 3^b + \beta_i \cdot x_{r_i} 3^{r_i}$.
	
	First, consider the case $c \neq r_i$. Thus, $c \notin R$.
	Observe that $x_b = y_b$, for all $b > c$ such that $b \notin R$.
	Furthermore, $x_c = 0$ and $y_c = 1$. Thus,
	$f_{\bT}(y) - f_{\bT}(x) > 3^c - \sum_{b < c} 3^b > 0$.
	An identical argument holds for $g_{\bT}$.
	
	Now, consider the case $c = r_i$. Thus,
	$f_{\bT}(y) - f_{\bT}(x) = \alpha_c 3^c + \sum_{b < c, b \notin R}
	(y_b - x_b) 3^b$. Using the same geometric series arguments
	as above, $\sgn_{f_{\bT}}(x,y) = \alpha_c$.
	By an analogous argument, we can show that $\sgn_{g_{\bT}}(x,y) = \beta_i$.
\end{proof}

\subsection{Proving Theorem~\ref{thm:tv}: Good and Bad Events} \label{sec:thm-tv-proof}
For a given \new{set of queries} $Q$, we first identify certain ``bad'' values for $\bS$,
%on which $Q$ could potentially distinguish between $f_{\bS}$ and $g_{\bS}$.
on which $Q$ could potentially distinguish between \new{$f_{\bT}$ and $g_{\bT}$ for any $\bT$ that extends $\bS$}.
We will
prove that the probability
of a bad $\bS$ is small for a given $Q$.
Furthermore, we show that $Q$ cannot distinguish
%between $f_{\bS}$ and $g_{\bS}$ for any good $\bS$.
between \new{$f_{\bT}$ and $g_{\bT}$ for any $\bT$ that extends good $\bS$}.
First, we set up some definitions.

\begin{definition} \label{def:cap}
	Given a pair $(x,y)$, define $\capt(x,y)$ to be the 5 most significant coordinates\footnote{There is nothing special about the constant $5$. It just needs to be sufficiently large.}
	in which they differ.
We say $(x,y)$ {\em captures} these coordinates.
For any set \new{of points} $S\subseteq \{0,1\}^{\dd}$, define $\capt(S) := \bigcup_{x,y \in S} \capt(x,y)$
to be the coordinates captured by the set $S$.
\end{definition}

\noindent Fix any $Q$. We set $Q_i := Q \cap C_i$.
We define two bad events for $\bS$.
\begin{itemize}
	\item Abort Event $\cA$: There exist $x,y \in Q$ with
    $\capt(x,y) \subseteq R$.
	\item Collision Event $\cC$: There exist $i,j \in [d]$
     with $r_i = r_j$, \new{such that} $r_i\in \capt(Q_i)$ and $r_j\in \capt(Q_j)$.
\end{itemize}

\noindent If $\cA$ does not occur, then for any pair $(x,y)$, the sign $\sgn_h(x,y)$ is determined by $\capt(x,y)$ for any $h\in \supp(\Yes)\cup \supp(\No)$.
The heart of the analysis lies in \Thm{bad}, which states that the bad events happen rarely.
\Thm{bad} is proved in~\Sec{bad}.

\begin{theorem} \label{thm:bad} If $|Q| \leq \delta d\log d$,
then $\Pr[\cA \cup \cC] = O(\delta)$.
\end{theorem}

\noindent When neither the abort nor the collision events happen,
we say $\bS$ is good for $Q$.
Next, we show that conditioned on a good $\bS$,
the set $Q$ cannot distinguish $f \sim \Yes$ from $g \sim \No$.

\begin{lemma}\label{lem:zero}
For any signed graph $G$ over $Q$,
$$\Pr_{f\sim \Yes} [G^Q_f = G|\bS \textrm{ is good}] \!=\! \Pr_{g\sim \No} [G^Q_g = G|\bS \textrm{ is good}].$$
\end{lemma}
\begin{proof}
We first describe the high level ideas in the proof. As stated above, when the abort event does not happen, the sign $\sgn_h(x,y)$ is determined by $\capt(x,y)$ for any $h\in \supp(\Yes)\cup \supp(\No)$.
Furthermore,  a pair $(x,y)$ has a possibility of distinguishing (that is, the pair is interesting) only if $x,y \in C_i$ and $r_i \in \capt(x,y)$.
Focus on such interesting pairs. For such a pair, both $\sgn_{f_{\bT}}(x,y)$ and $\sgn_{g_{\bT}}(x,y)$ are equally likely to be $+1$ or $-1$.
Therefore, to distinguish, we would need two interesting pairs, $(x,y) \in C_i$ and $(x',y') \in C_j$ with $i \neq j$. Note that, when $g \sim \No$,
the signs $\sgn_{g_{\bT}}(x,y)$ and $\sgn_{g_{\bT}}(x',y')$ are independently set, whereas when $f \sim \Yes$, the signs are either the same when $r_i = r_j$, or independently set.
But if the collision event does not occur, then $r_i \neq r_j$ for interesting pairs in different subcubes. Therefore, the probabilities are the same.

Now, we prove the lemma formally.
	Condition on a good $\bS$. Note that
	the probability of the $\Yes$ distribution depends
	solely on $\{\alpha_b\}$ and that of the $\No$ distribution
	depends solely on $\{\beta_i\}$.
	
	Consider any pair $(x,y) \in Q\times Q$ with $x \prec y$.
	We can classify it into three types: (i) $x$ and $y$ are in different
	subcubes, (ii) $x$ and $y$ are both in the same subcube $C_i$, and $\coord{i}{\bS}(x,y) \neq r_i$,
	(iii) $x$ and $y$ are both in $C_i$, and $\coord{i}{\bS}(x,y) = r_i$.
	For convenience, we refer to the third type as {\em interesting pairs}.
	Let $h \in \supp(\Yes | \bS) \cup \supp(\No | \bS)$.
	For the first and second types of pairs, by \Clm{inv} and \Clm{interesting}, we have $\sgn_h(x,y) = 1$.
	For interesting pairs, by \Clm{interesting}, $\sgn_h(x,y)$
	must have the same label for all pairs in $Q_i \times Q_i$.
	Thus, any $G$ whose labels disagree with the above can never
	be $G^Q_f$ or $G^Q_g$.
	
	Fix a signed graph $G$. For any pair $(x,y) \in Q \times Q$, where $x \prec y$,
	let $w(x,y)$ be the label in $G$. Furthermore, for all interesting pairs
	%in the same $Q_i$, $w(x,y)$ has the same label, denoted $w_i$.
	\new {within the same $Q_i$, the label $w(x,y)$ is the same and denoted by $w_i$.}
	Let $I$ denote the set of subcubes with interesting pairs.
	At this point, all of our discussion depends purely on $\bS$
	and involves no randomness.
	
	Now we focus on $g \sim (\No|\bS)$.
	\begin{eqnarray*}
		\Pr_{g\sim (\No|\bS)}[G^Q_g = G]
		& = & \Pr\Big[\bigwedge_{i \in I} \bigwedge_{\substack{x,y \in Q_i \\ \coord{i}{\bS}(x,y) = r_i}}
		(w(x,y) = \sgn_{g_{\bT}}(x,y))\Big]  \\
		& = & \Pr\Big[\bigwedge_{i \in I} \bigwedge_{\substack{x,y \in Q_i \\ \coord{i}{\bS}(x,y) = r_i}}
		(w(x,y) = \beta_{i})\Big]  \ \ \ \textrm{(by \Clm{interesting})}\\
		& = & \Pr\Big[\bigwedge_{i \in I} (w_i = \beta_{i})\Big].
	\end{eqnarray*}
	
	\noindent Observe that each $\beta_{i}$ is chosen uniformly and independently at random from $\{-1,+1\}$, and so this 	probability is exactly $2^{-|I|}$.
	
%	The analogous expressions for $f \sim (\Yes|\bS)$ yield:
\new{The analogous expression for $f \sim (\Yes|\bS)$ yields}
	$$ \Pr_{f\sim (\Yes|\bS)}[G^Q_f = G] = \Pr \Big[\bigwedge_{i \in I} (w_i = \alpha_{r_i})\Big] .$$
	\new{Notice that if multiple $r_i$'s are the same, then the individual events are not independent over different subcubes.}
%	Note the difference here: if multiple $r_i$'s are the same, the individual events
%	are not independent over different subcubes. 
This is precisely what the abort and collision
	events capture. We formally argue below.
	
	Consider an interesting pair $(x,y) \in Q_i \times Q_i$. Since the abort event $\cA$ does
	not happen, $\capt(x,y) \nsubseteq R$. If $\coord{i}{\bS}(x,y) = r_i \notin \capt(x,y)$,
	then there is a coordinate of $\overline{R}$ that is more significant than $\coord{i}{\bS}(x,y)$.
	This contradicts the definition of the latter; so $r_i \in \capt(x,y) \subseteq \capt(Q_i)$.
	Equivalently, a subcube index $i \in I$ iff $r_i \in \capt(Q_i)$.
	
	Since the collision event $\cC$ does not happen, for any $j \in [m]$ such that $r_j = r_i$, we have $r_j \notin \capt(Q_j)$. 
	Alternately, for any $i,i' \in I$,
 we have $r_i \neq r_{i'}$. Thus, $\Pr[\bigwedge_{i \in I}(w_i = \alpha_{r_i})]
	= \prod_{i \in I}\Pr[w_i = \alpha_{r_i}] = 2^{-|I|}$,
	\new{completing the proof of the lemma.}
\end{proof}

\noindent Now, we are armed to prove \Thm{tv}.

\begin{proof}[Proof of \Thm{tv}]
	Given any subset of signed graphs, $\calG$, it suffices to upper bound
	\begin{align*}
	\left|\Pr_{f\sim \Yes} [G^Q_f \in \calG] - \Pr_{f\sim \No} [G^Q_f \in \calG]\right|  
	&\leq 
    \sum_{\textrm{good } \bS} \left|\Pr[\bS]\cdot\left(\Pr_{f\sim \Yes} [G^Q_f \in \calG | \bS] - \Pr_{f\sim \No} [G^Q_f \in \calG|\bS]\right) \right| \\
    & +  \sum_{\textrm{bad } \bS} \left|\Pr[\bS]\cdot\left(\Pr_{f\sim \Yes} [G^Q_f \in \calG|\bS] - \Pr_{f\sim \No} [G^Q_f \in \calG|\bS]\right) \right|.
	\end{align*}
	The first term of the RHS is $0$ by \Lem{zero}.
    The second term is at most the probability of bad events, which is $O(\delta)$
    by \Thm{bad}.
\end{proof}

\subsection{Bounding the Probability of Bad Events: Proof of Theorem~\ref{thm:bad}}\label{sec:bad}

We prove \Thm{bad} by individually bounding $\Pr[\calA]$ and $\Pr[\calC]$.

\begin{lemma}\label{lem:A}
	If $|Q|\leq \delta d\log d$, then $\Pr[\calA] \leq d^{-1/4}$.
\end{lemma}

\begin{proof} Fix any choice of $k$ (in $\bS$).
For any pair of points $x,y \in Q$, we have $\Pr[\capt(x,y) \subseteq R] \leq (\frac{2^k}{d-5})^5$.  Since $d-5 \geq d/2$ for all $d \geq 10$ and $k \leq (\log_2d)/2$, the probability is at most $32d^{-5/2}$.
\new{By a union bound, $\Pr[\cA] \leq |Q \times Q| \cdot 32d^{-5/2} \leq d^{-1/4}$ for a large enough $d$}.
%For a large enough $d$, a union
%bound over all pairs in $Q \times Q$, which are at most $d^2\log^2d$ in number, completes the proof.
\end{proof}

\noindent 
\new{The most challenging part of this work is bounding the probability of the collision event, which forms the heart of the lower bound.}
%The collision event is more challenging to bound.
%Bounding it is the heart of the lower bound.
We start by showing that, if each $Q_i$ captures
few coordinates, then the collision event has low probability. A critical point is the appearance of $d\log d$ in this bound.

\begin{lemma}\label{lem:prob}
If $\sum_i |\capt(Q_i)| \leq M$, then $\Pr[\cC] = O\left(\frac{M}{d\log d}\right)$.
\end{lemma}

\begin{proof} 
	For any $r \in [d]$, 
	define $A_r := \{j: r\in \capt(Q_j)\}$ to be the set of indices of $Q_j$'s that capture coordinate $r$.
    Let $a_r := |A_r|$. Define $n_\ell := |\{r: a_r \in (2^{\ell-1},2^{\ell}]\}|$.
	Observe that $\sum_{\ell \leq \log_2 d} n_\ell 2^\ell \leq 2\sum_{r\in [d]} a_r \leq 2M$.
	
	Fix $k$. For $r\in [d]$, we say the event $\cC_r$ occurs if
	(a) $r \in R$, and (b) there exists $i,j\in [d]$ such that $r_i = r_j = r$, and $r_i \in \capt(Q_i)$ and $r_j\in \capt(Q_j)$. By the union
    bound, $\Pr[\cC| k] \leq \sum_{r=1}^d \Pr[\cC_r | k]$.
	
	Now, we compute $\Pr[\cC_r|k]$. 
	Only sets $Q_j$'s with $j \in A_r$ are of interest, since the others do not capture $r$.
	Event $\cC_r$ occurs if at least two of these sets have $r_i = r_j = r$. Hence,
\begin{align}
\Pr[\cC_r|k] & = \Pr[r\in R]\cdot \Pr[\exists i,j\in A_r: r_i = r_j = r ~|~r\in R] \notag \\
& = \frac{2^{k}}{d}\cdot \sum_{c \geq 2} {a_r \choose c} \left(\frac{1}{2^k} \right)^c \left(1-\frac{1}{2^k} \right)^{a_r-c}. \label{eq:007}
\end{align}
A fixed $r$ is in $R$ with probability ${d-1 \choose 2^k-1}/{d\choose 2^k} = \frac{2^k}{d}$.
Given that $|R| = 2^k$, the probability that $r_i = r$ is precisely $2^{-k}$.

	If $a_r \geq \frac{2^k}{4}$, then we simply upper bound \eqref{eq:007} by  $\frac{2^k}{d}$. 
	For $a_r < \frac{2^k}{4}$, %the summation
	we upper bound \eqref{eq:007} by
\begin{align*}
	\frac{2^k}{d} \left(1-\frac{1}{2^k}\right)^{a_r} \sum_{c\geq 2} \left( a_r \cdot \frac{1}{2^k} \cdot \left( 1-\frac{1}{2^k} \right)^{-1} \right)^c \leq \frac{2^k}{d} \sum_{c \geq 2} \left( \frac{a_r}{2^{k-1}} \right)^c \leq \frac{8a^2_r}{2^k d}.
\end{align*}
Summing over all $r$ and grouping according
to $n_\ell$, we get
\begin{align*}
\Pr[\cC|k] \leq \sum_{r=1}^d \Pr[\cC_r|k] 
\leq \sum_{r: a_r \geq 2^{k-2}} \frac{2^k}{d} + \frac{8}{d} \sum_{r: a_r < 2^{k-2}} \frac{a^2_r}{2^k} 
\leq \frac{2^k}{d} \sum_{\ell > k-2} n_\ell + \frac{8}{d}  \sum_{\ell=1}^{k-2} n_\ell 2^{2\ell - k} .
\end{align*}
Averaging over all $k$, we get
\begin{align}
	\Pr[\cC] & = \frac{2}{\log_2 d} \sum_{k=1}^{(\log_2 d)/2}\Pr[\cC | k]   \quad \leq \quad \frac{16}{d \log_2 d} \sum_{k=1}^{(\log_2 d)/2} \left( \sum_{\ell=1}^{k-2} n_\ell 2^{2\ell - k} + \sum_{\ell > k-2} n_\ell 2^k \right) \notag \\
	& = \frac{16}{d\log_2 d} \left(\sum_{\ell=1}^{(\log_2 d)/2} n_\ell \sum_{k \geq \ell + 2} 2^{2\ell - k} + \sum_{\ell=1}^{\log_2 d} n_\ell \sum_{k < \ell+2} 2^k  \right).\label{eq:008}
\end{align}

\noindent
Now, $\sum_{k\geq \ell+2} 2^{2\ell - k} \leq 2^\ell$ and $\sum_{k < \ell+2} 2^k \leq 4\cdot 2^\ell$. Substituting,
$
\Pr[\cC] \leq \frac{80}{d\log_2 d} \sum_{\ell=1}^{\log_2 d} n_\ell 2^\ell \leq \frac{160M}{d\log_2 d}
$, proving the lemma.
\end{proof}

\noindent We are now left to bound $\sum_i |\capt(Q_i)|$. This is done by the following combinatorial lemma.

\begin{lemma}\label{lem:comb}
Let $V$ be a set of vectors over an arbitrary alphabet
and any number of dimensions. For any natural number $c$
and $x,y \in V$,
let $\capt_c(x,y)$ denote the (set of) first $c$ coordinates
at which $x$ and $y$ differ. Then $|\capt_c(V)| \leq c(|V|-1)$.
\end{lemma}

\begin{proof}
We construct $c$ different edge-colored graphs $G_1, \ldots, G_c$ over the vertex set $V$. 
For every coordinate $i\in \capt_c(V)$, there must exist at least one pair of
vectors $x,y$ such that $i \in \capt_c(x,y)$. Thinking
of each $\capt_c(x,y)$ as an ordered set, find a pair
$(x,y)$ where $i$ appears ``earliest'' in $\capt_c(x,y)$.
Let the position of $i$ in this $\capt_c(x,y)$ be denoted by $t$.
We add edge $(x,y)$ to $G_t$, and color it $i$.
Note that the same edge $(x,y)$ cannot be added to $G_t$
with multiple colors, and hence all $G_t$'s are simple graphs.
Furthermore, observe that each color is present only
once over all $G_t$'s.

We claim that each $G_t$ is acyclic. Suppose not. Let there be a cycle $C$ and let $(x,y)$ be the edge in $C$ with the smallest color $i$. Clearly, $x_i \neq y_i$ since $i \in \capt_c(x,y)$. There must exist another edge $(u,v)$ in $C$
such that $u_i \neq v_i$. Furthermore, the color of $(u,v)$
is $j > i$. Thus, $j$ is the $\ord{t}$ entry in $\capt_c(u,v)$.
Note that $i \in \capt_c(u,v)$ and must be the $\ord s$ entry
for some $s < t$. But this means that the edge $(u,v)$
colored $i$ should be in $G_s$, contradicting
the presence of $(x,y) \in G_t$.
\end{proof}

\noindent We wrap up the bound now.

\begin{lemma}\label{lem:C}
	If $|Q|\leq \delta d\log d$, then $\Pr[\calC] = O(\delta)$.
\end{lemma}
\begin{proof}
\Lem{comb} applied to each $Q_i$, yields
$\sum_i |\capt(Q_i)| \leq 5|Q_i| = 5|Q|$.  An application of
\Lem{prob} completes the proof.
\end{proof}

\chapter{Conclusion}

\section{Conclusion and Open Directions}
In this work, we give the first algorithms for 
testing unateness of real-valued functions over the hypercube as well as the hypergrid domains.
We also show that our algorithms are optimal by proving matching lower bounds, thus resolving the query complexity of testing unateness of real-valued functions.
Our results demonstrate that, for real-valued functions, in contrast to monotonicity testing, adaptivity helps with testing unateness. 

%We completed the problem of testing Unateness of the real-valued functions
The query complexity of testing unateness of Boolean functions has not been completely resolved yet.
Concurrent with our work, Chen et al.~\cite{CWX17} proved a lower bound of $\Omega\left(\frac{d^{2/3}}{\log^3 d} \right)$ for adaptive unateness testers of Boolean functions over $\{0,1\}^d$.
Subsequently, Chen et al.~\cite{CWX17focs} gave an adaptive unateness tester with query complexity $\widetilde{O}\left(\frac{d^{3/4}}{\eps^2}\right)$ for the same class of functions.
It remains to be seen if this polynomial gap (in $d$) between the bounds can be closed further.

On nonadaptive unateness testing of Boolean functions over $\{0,1\}^d$, in a subsequent work, Baleshzar et al.~\cite{BCPRS17b} proved a lower bound of $\Omega\left(\frac{d}{\log d} \right)$ for one-sided error testers.
Since Boolean functions are special cases of real-valued functions, 
our nonadaptive algorithm over the hypercube also works for Boolean functions.
The query complexity of this algorithm is $O\left(\frac{d \log d}{\eps}\right)$ which is currently the best known upper bound.
An interesting open question is to determine if testers with two-sided error have better query complexity than testers with one-sided error in the nonadaptive setting. 
%but there are still some gaps for the boolean case. According to this work the best known nonadaptive upper bound for testing unateness of boolean function over the hypercube domain is $O(\frac{d\log d}{\eps})$, even though the best known lower bound according to a subsequent work to this work~\cite{GGLRS00} is $\Omega(\frac{d}{\log d})$. Also, for the adaptive case we show the upper bound of $O(\frac{d}{\eps})$ in this work while the best known lower bound is $\Omega(\frac{\sqrt{d}}{\log^2 d})$ according to~\cite{GGLRS00}.
\chapter{Title of the Fifth Chapter}

\section{Introduction}
When in the Course of human events, it becomes necessary for one people  to dissolve the political bands which have connected them with another,  and to assume among the powers of the earth, the separate and equal station  to which the Laws of Nature and of Nature's God entitle them, a decent respect to the opinions of mankind requires that they should declare  the causes which impel them to the separation.

\section{More Declaration}

We hold these truths to be self-evident, that all men are created equal,  that they are endowed by their Creator with certain unalienable Rights,  that among these are Life, Liberty and the pursuit of Happiness. --That to secure these  rights, Governments are instituted among Men, deriving their just powers  from the consent of the governed, --That whenever any Form of Government  becomes destructive of these ends, it is the Right of the People to alter  or to abolish it, and to institute new Government, laying its foundation on  such principles and organizing its powers in such form, as to them shall  seem most likely to effect their Safety and Happiness. Prudence, indeed, will dictate that Governments long established should not  be changed for light and transient causes; and accordingly all experience  hath shewn, that mankind are more disposed to suffer, while evils are  sufferable, than to right themselves by abolishing the forms to which they  are accustomed. But when a long train of abuses and usurpations, pursuing invariably the same  Object evinces a design to reduce them under absolute Despotism, it is their  right, it is their duty, to throw off such Government, and to provide new Guards for their future security. --Such has been the patient sufferance of these Colonies; and such is now the  necessity which constrains them to alter their former Systems of Government.  The history of the present King of Great Britain [George III] is a history  of repeated injuries and usurpations, all having in direct object the  establishment of an absolute Tyranny over these States. To prove this, let Facts be submitted to a candid world.
%%%%%%%%%%%%%%%%%%%%%%%%%%%%%%%%%%%%%%%%%%%%%%%%%%%%%%%%%%%%%%%
% Appendices
%
% Because of a quirk in LaTeX (see p. 48 of The LaTeX
% Companion, 2e), you cannot use \include along with
% \addtocontents if you want things to appear the proper
% sequence.
%%%%%%%%%%%%%%%%%%%%%%%%%%%%%%%%%%%%%%%%%%%%%%%%%%%%%%%%%%%%%%%
\appendix
\titleformat{\chapter}[display]{\fontsize{30}{30}\selectfont\bfseries\sffamily}{Appendix \thechapter\textcolor{gray75}{\raisebox{3pt}{|}}}{0pt}{}{}
% If you have a single appendix, then to prevent LaTeX from
% calling it ``Appendix A'', you should uncomment the following two
% lines that redefine the \thechapter and \thesection:
%\renewcommand\thechapter{}
%\renewcommand\thesection{\arabic{section}}
\Appendix{Missing Details from the Main Body}

\section{The Lower Bound for Adaptive Testers over Hypergrids} \label{sec:adap-lb}
We show that every unateness tester for functions $f:[n]^d \mapsto \R$  requires $\Omega\left(\frac{d \log n}{\eps}-\frac{\log 1/\eps}{\eps}\right)$ queries for $\eps\in(0,1/4)$ and prove Theorem~\ref{thm:adap-lb}.

\begin{proof}[Proof of Theorem~\ref{thm:adap-lb}]
By Yao's minimax principle and the reduction to testing with comparison-based testers from~\cite{Fis04,CS14} (stated for completeness in Theorem~\ref{thm:CS14}), it is sufficient to give a hard input distribution on which every deterministic comparison-based tester fails with probability more than 2/3. We use the hard distribution constructed  by Chakrabarty and Seshadhri~\cite{CS14} to prove the same lower bound for testing monotonicity. Their distribution is a mixture of two distributions, $\Yes$ and $\No,$ on positive and negative instances, respectively. Positive instances for their problem are functions that are monotone and, therefore, unate; negative instances are functions that are $\eps$-far from monotone. We show that their $\No$ distribution is supported on functions that are $\eps$-far from unate, i.e., negative instances for our problem. Then the required lower bound for unateness follows from the fact that every deterministic comparison-based tester needs the stated number of queries to distinguish $\Yes$ and $\No$ distributions with high enough probability.

We start by describing the $\Yes$ and $\No$ distribution used in \cite{CS14}. We will define them as distributions on functions over the hypercube domain. Next, we explain how to convert functions over hypercubes  to functions over hypergrids.

Without loss of generality, assume $n$ is a power of $2$ and let $\ell := \log_2 n$.
% Assume $[n]$ to be $\{0,1,\ldots, n-1\}$.
For any $z \in [n]$, let $bin(z)$ denote the binary representation of $z-1$ as an $\ell$-bit vector $(z_1, \ldots, z_\ell)$, where $z_1$ is the least significant bit.

We now describe the mapping used to convert functions on hypergrids to functions on hypercubes. Let $\phi:[n]^d \to \{0,1\}^{d\ell}$ be the mapping that takes $y\in[n]^d$ to the concatenation of $bin(y_1),\dots,bin(y_d)$. Any function $f:\{0,1\}^{d \ell} \mapsto \R$ can be easily converted into a function $\widetilde{f}:[n]^d \mapsto \R$, where $\widetilde{f}(y) := f(\phi(y))$.

Let $m:= d\ell$. For $x \in \{0,1\}^m$, let $\texttt{val}(x) = \sum\nolimits_{i=1}^m x_i 2^{i-1}$ denote the value of the binary number represented by vector $x$. For simplicity, assume $1/\eps$ is a power of $2$.
Partition the set of points $x \in \{0,1\}^m$ according to the most significant $\log(1/2\eps)$ dimensions.
That is, for $k \in \{1,2,\ldots, 1/2\eps\}$, let
$$S_k := \{x: \texttt{val}(x) \in [(k-1)\cdot \eps 2^{m+1}, k\cdot\eps 2^{m+1} - 1]\}.$$
The hypercube is partitioned into $1/2\eps$ sets $S_k$ of equal size, and each $S_k$ forms a subcube of dimension $m' = m - \log(1/\eps) + 1$.

We now describe the $\Yes$ and $\No$ distributions for functions on hypercubes.
The $\Yes$ distribution consists of a single function $f(x) = 2 \texttt{val}(x)$.
The $\No$ distribution is uniform over $m'/2\eps$ functions $g_{j,k}$, where $j \in [m']$ and $k \in [1/2\eps]$, defined as follows:
\begin{align*}
    g_{j,k}(x) =
    \begin{cases}
        2\texttt{val}(x) - 2^j - 1 &\text{ if } x_j = 1 \text{ and } x \in S_k;\\
        2\texttt{val}(x), &\text{ otherwise.}
    \end{cases}
\end{align*}
To get the $\Yes$ and $\No$ distributions for the hypergrid, we convert $f$ to $\widetilde{f}$ and each function $g_{j,k}$ to $\widetilde{g_{j,k}}$, using the transformation defined before.


Chakrabarty and Seshadhri~\cite{CS14} proved that $f$ is monotone and each function $\widetilde{g_{j,k}}$  is $\eps$-far from monotone. It remains to show that functions $\widetilde{g_{j,k}}$ are also $\eps$-far from unate.

\begin{claim}
Each function $\widetilde{g_{j,k}}$ is $\eps$-far from unate.
\end{claim}

\begin{proof}
To prove that $\widetilde{g_{j,k}}$ is $\eps$-far from unate, it suffices to show that there exists a dimension $i$, such that there are at least $\eps 2^{d\ell}$ increasing $i$-pairs and at least $\eps 2^{d\ell}$ decreasing $i$-pairs w.r.t.\ $\widetilde{g_{j,k}}$ and that all of these $i$-pairs are disjoint. Let $u,v \in [n]^d$ be two points such that $\phi(u)$ and $\phi(v)$ differ only in the $\ord{j}$ bit. Clearly, $u$ and $v$ form an $i$-pair, where $i =\lceil j/\ell \rceil$. Now, if $\phi(u),\phi(v) \in S_k$ and $u \prec v $, then $\widetilde{g_{j,k}}(v) = \widetilde{g_{j,k}}(u) - 1$. So, the $i$-pair $(u,v)$ is decreasing. The total number of such $i$-pairs is  $2^{d\ell - \log(1/2\eps) - 1} = \eps 2^{d\ell}$. If $\phi(u),\phi(v) \in S_{k'}$ where $k' \neq k$, then the $i$-pair $(u,v)$ is increasing. Clearly, there are at least $\eps 2^{d\ell}$ such $i$-pairs. All the $i$-pairs we mentioned are disjoint.
Hence, $\widetilde{g_{j,k}}$ is $\eps$-far from unate.
\end{proof}
This completes the proof of Theorem~\ref{thm:adap-lb}.
\end{proof}

\section{The Lower Bound for Nonadaptive Testers over Hypergrids}\label{sec:na-lb-hg}
The lower bound for nonadaptive testers over hypergrids follows from a combination of the lower bound for nonadaptive testers over hypercube and the lower bound for adaptive testers over hypergrids.

\begin{theorem}
Any nonadaptive unateness tester (even with two-sided error) for real-values functions $f:[n]^d \mapsto \R$ must make $\Omega(d(\log n + \log d))$ queries.
\end{theorem}
\begin{proof}
Fix $\eps = 1/8$. The proof consists of two parts. The lower bound for adaptive testers is also a lower bound for nonadaptive tester, and so, the bound of $\Omega(d \log n)$ holds.
Next, we extend the $\Omega(d \log d)$ lower bound for hypercubes.
Assume $n$ to be a power of $2$. Define function $\psi:[n] \mapsto \{0,1\}$ as $\psi(a):= \mathbbm{1}[a > n/2]$ for $a \in [n]$. For $x = (x_1, x_2, \ldots, x_d) \in [n]^d$, define the mapping $\Psi: [n]^d \mapsto \{0,1\}^d$ as $\Psi(x) := (\psi(x_1), \psi(x_2), \ldots, \psi(x_d))$. Any function $f:\{0,1\}^d \mapsto \R$ can be extended to $\tilde{f}:[n]^d \mapsto \R$ using the mapping $\tilde{f}(x) = f(\Psi(x))$ for all $x \in [n]^d$. The proof of Theorem~\ref{thm:non-adap-lb} goes through for hypergrids as well, and so we have an $\Omega(d \log d)$ lower bound. Combining the two lower bounds, we get a bound of $\Omega(d \cdot \max\{\log n, \log d\})$, which is asymptotically equal to $\Omega(d(\log n + \log d))$.
\end{proof}
\end{document}

\Appendix{Title of the Second Appendix}

\section{Introduction}
When in the Course of human events, it becomes necessary for one people  to dissolve the political bands which have connected them with another,  and to assume among the powers of the earth, the separate and equal station  to which the Laws of Nature and of Nature's God entitle them, a decent respect to the opinions of mankind requires that they should declare  the causes which impel them to the separation.

\section{More Declaration}

We hold these truths to be self-evident, that all men are created equal,  that they are endowed by their Creator with certain unalienable Rights,  that among these are Life, Liberty and the pursuit of Happiness. --That to secure these  rights, Governments are instituted among Men, deriving their just powers  from the consent of the governed, --That whenever any Form of Government  becomes destructive of these ends, it is the Right of the People to alter  or to abolish it, and to institute new Government, laying its foundation on  such principles and organizing its powers in such form, as to them shall  seem most likely to effect their Safety and Happiness. Prudence, indeed, will dictate that Governments long established should not  be changed for light and transient causes; and accordingly all experience  hath shewn, that mankind are more disposed to suffer, while evils are  sufferable, than to right themselves by abolishing the forms to which they  are accustomed. But when a long train of abuses and usurpations, pursuing invariably the same  Object evinces a design to reduce them under absolute Despotism, it is their  right, it is their duty, to throw off such Government, and to provide new Guards for their future security. --Such has been the patient sufferance of these Colonies; and such is now the  necessity which constrains them to alter their former Systems of Government.  The history of the present King of Great Britain [George III] is a history  of repeated injuries and usurpations, all having in direct object the  establishment of an absolute Tyranny over these States. To prove this, let Facts be submitted to a candid world.
\Appendix{Title of the Third Appendix}

\section{Introduction}
When in the Course of human events, it becomes necessary for one people  to dissolve the political bands which have connected them with another,  and to assume among the powers of the earth, the separate and equal station  to which the Laws of Nature and of Nature's God entitle them, a decent respect to the opinions of mankind requires that they should declare  the causes which impel them to the separation.

\section{More Declaration}

We hold these truths to be self-evident, that all men are created equal,  that they are endowed by their Creator with certain unalienable Rights,  that among these are Life, Liberty and the pursuit of Happiness. --That to secure these  rights, Governments are instituted among Men, deriving their just powers  from the consent of the governed, --That whenever any Form of Government  becomes destructive of these ends, it is the Right of the People to alter  or to abolish it, and to institute new Government, laying its foundation on  such principles and organizing its powers in such form, as to them shall  seem most likely to effect their Safety and Happiness. Prudence, indeed, will dictate that Governments long established should not  be changed for light and transient causes; and accordingly all experience  hath shewn, that mankind are more disposed to suffer, while evils are  sufferable, than to right themselves by abolishing the forms to which they  are accustomed. But when a long train of abuses and usurpations, pursuing invariably the same  Object evinces a design to reduce them under absolute Despotism, it is their  right, it is their duty, to throw off such Government, and to provide new Guards for their future security. --Such has been the patient sufferance of these Colonies; and such is now the  necessity which constrains them to alter their former Systems of Government.  The history of the present King of Great Britain [George III] is a history  of repeated injuries and usurpations, all having in direct object the  establishment of an absolute Tyranny over these States. To prove this, let Facts be submitted to a candid world.
\Appendix{Title of the Fourth Appendix}

\section{Introduction}
When in the Course of human events, it becomes necessary for one people  to dissolve the political bands which have connected them with another,  and to assume among the powers of the earth, the separate and equal station  to which the Laws of Nature and of Nature's God entitle them, a decent respect to the opinions of mankind requires that they should declare  the causes which impel them to the separation.

\section{More Declaration}

We hold these truths to be self-evident, that all men are created equal,  that they are endowed by their Creator with certain unalienable Rights,  that among these are Life, Liberty and the pursuit of Happiness. --That to secure these  rights, Governments are instituted among Men, deriving their just powers  from the consent of the governed, --That whenever any Form of Government  becomes destructive of these ends, it is the Right of the People to alter  or to abolish it, and to institute new Government, laying its foundation on  such principles and organizing its powers in such form, as to them shall  seem most likely to effect their Safety and Happiness. Prudence, indeed, will dictate that Governments long established should not  be changed for light and transient causes; and accordingly all experience  hath shewn, that mankind are more disposed to suffer, while evils are  sufferable, than to right themselves by abolishing the forms to which they  are accustomed. But when a long train of abuses and usurpations, pursuing invariably the same  Object evinces a design to reduce them under absolute Despotism, it is their  right, it is their duty, to throw off such Government, and to provide new Guards for their future security. --Such has been the patient sufferance of these Colonies; and such is now the  necessity which constrains them to alter their former Systems of Government.  The history of the present King of Great Britain [George III] is a history  of repeated injuries and usurpations, all having in direct object the  establishment of an absolute Tyranny over these States. To prove this, let Facts be submitted to a candid world.
%\Appendix{Title of the Fifth Appendix}

\section{Introduction}
When in the Course of human events, it becomes necessary for one people  to dissolve the political bands which have connected them with another,  and to assume among the powers of the earth, the separate and equal station  to which the Laws of Nature and of Nature's God entitle them, a decent respect to the opinions of mankind requires that they should declare  the causes which impel them to the separation.

\pagebreak
Some text.
{\lstset{language=Fortran}
\footnotesize
\begin{lstlisting}
      program chaos
c When a LS Fortran program has been compiled and linked into Mac
c application, all information written to the screen WRITE(6,...) or
c WRITE(*,...) appears in a standard Mac window, complete with basic
c menus.
      external fex, jac
      double precision atol, rtol, rwork, t, tout, h
      double precision ttotal, dtout
      dimension h(3), atol(3), rwork(70), iwork(23)
	  character*8 tstart, tend
      neq = 3
	  
	  call time(tstart)
	  write(6,*) "begin integration at  ", tstart
      write(6,*)
	  
c --- Read in the total initial angular momentum.  The total angular
c     momentum H is always unity due to normalization.
	  open(unit = 2, file = 'chaos.data', status = 'unknown')
      read(2,*) h(1), h(2), h(3)
	  
c --- The integration begins at t = 0 and the values are printed at
c     every tout.  tout is incremented below.  ttotal is the length
c     of the entire integration.  The number of recorded values of
c     the integration is given by npoints.
      t = 0.0d0
      tout = 0.0d0
      write(6,*) 'Duration of integration interval, i.e., tfinal?'
      read(6,*) ttotal
      write(6,*)
      write(6,*) 'Number of points for trajectory plot?'
      read(6,*) npoints
      write(6,*)
      dtout = ttotal/dfloat(npoints)
      tout = tout + dtout
	  
c --- Tolerance parameters used by lsoda.
      itol = 2
      rtol = 1.0d-9
      atol(1) = 1.0d-9
      atol(2) = 1.0d-9
      atol(3) = 1.0d-9
	  
c --- Other parameters used by lsoda.  See below.
      itask = 1
      istate = 1
      iopt = 1
      lrw = 70
      liw = 23
      jt = 1

      do 11 kount = 5,10
         rwork(kount) = 0.0d0
         iwork(kount) = 0
  11  continue
      iwork(6) = 100000
	  
	  open(unit = 3, file = 'traj.dat', disp = 'keep',
     &     status = 'unknown')
	 
c --- The actual integration begins here.  Loop on the value of iout.
      do 40 iout = 1, npoints
	  
         call lsoda(fex,neq,h,t,tout,itol,rtol,atol,itask,istate,
     &              iopt,rwork,lrw,iwork,liw,jdum,jt)
	  
c ------ Write the output to the file traj.dat.
         write(3,20) t, h(1), h(2), h(3)
  20     format(f9.1, 3e15.6)

         if (mod(tout,5000.0d0) .eq. 0.0d0) then
            write(6,*) tout
         end if
  
c ------ Check to see that things are going OK.
         if (istate .lt. 0) go to 80
		 
c ------ Set the time at which the integration is next recorded and
c        continue the do-loop.
  40     tout = tout + dtout
  
      write(6,*) 'number of steps taken: ', iwork(11)
      write(6,*) 'number of f evaluations: ', iwork(12)
      write(6,*) 'number of Jacobian evaluations: ', iwork(13)
      write(6,*) 'method order last used: ', iwork(14)
      write(6,*) 'method last used (2 = stiff): ', iwork(19)
      write(6,*) 'value of t at last method switch: ', rwork(15)
      write(6,*)
	 
	  call time(tend)
	  write(6,*) "end integration at  ", tend
      stop
	  
c --- If there is an error, given by istate < 0, write the following.
  80  write(6,90) istate
  90  format(///22h error halt.. istate =,i3)
  
      stop
      end

\end{lstlisting}
}

%%%%%%%%%%%%%%%%%%%%%%%%%%%%%%%%%%%%%%%%%%%%%%%%%%%%%%%%%%%%%%%
% ESM students need to include a Nontechnical Abstract as the %
% last appendix.                                              %
%%%%%%%%%%%%%%%%%%%%%%%%%%%%%%%%%%%%%%%%%%%%%%%%%%%%%%%%%%%%%%%
% This \include command should point to the file containing
% that abstract.
%\include{nontechnical-abstract}
%%%%%%%%%%%%%%%%%%%%%%%%%%%%%%%%%%%%%%%%%%%
} % End of the \allowdisplaybreak command %
%%%%%%%%%%%%%%%%%%%%%%%%%%%%%%%%%%%%%%%%%%%

%%%%%%%%%%%%%%%%
% BIBLIOGRAPHY %
%%%%%%%%%%%%%%%%
% You can use BibTeX or other bibliography facility for your
% bibliography. LaTeX's standard stuff is shown below. If you
% bibtex, then this section should look something like:
	\begin{singlespace}
	\bibliographystyle{GLG-bibstyle}
	\addcontentsline{toc}{chapter}{Bibliography}
	\bibliography{Biblio-Database}
	\end{singlespace}

%\begin{singlespace}
%\begin{thebibliography}{99}
%\addcontentsline{toc}{chapter}{Bibliography}
%\frenchspacing

%\bibitem{Wisdom87} J. Wisdom, ``Rotational Dynamics of Irregularly Shaped Natural Satellites,'' \emph{The Astronomical Journal}, Vol.~94, No.~5, 1987  pp. 1350--1360.

%\bibitem{G&H83} J. Guckenheimer and P. Holmes, \emph{Nonlinear Oscillations, Dynamical Systems, and Bifurcations of Vector Fields}, Springer-Verlag, New York, 1983.

%\end{thebibliography}
%\end{singlespace}

\backmatter

% Vita
\vita{SupplementaryMaterial/Vita}

\end{document}

